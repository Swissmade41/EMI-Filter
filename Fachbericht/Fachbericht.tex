\let\counterwithout\relax
\let\counterwithin\relax
\documentclass[final]{fhnwreport}       %[mode] = draft or final
\usepackage{color, colortbl}
\usepackage{rotating, rotfloat,ragged2e, hyphenat, diagbox, wrapfig}
\definecolor{grau}{gray}{0.9}
\definecolor{hellgrau}{gray}{0.95}

                                        %{class} = fhnwreport, article, 
                                        %          report, book, beamer, standalone
\input{header}			                %loads all packages, definitions and settings												
\title{«DJ» EMI Filter für Schaltnetzteil}          			%Project Title
\author{Fachbericht}  		%Document Type => Technical Report, ...
\date{Windisch, 13.06.2019}             		%Place and Date

\begin{document}

%%---TITLEPAGE---------------------------------------------------------------------------
\selectlanguage{ngerman}                %ngerman or english
\maketitle

\vspace*{-1cm}						    %compensates the space after the date line.
\vfill
\begin{figure}[H]
\centering
\includegraphics[width=10cm]{titelBild.png}
\end{figure}
\vfill

{
\renewcommand\arraystretch{2}
\begin{center}
\begin{tabular}{ >{\bf} l p{10cm} l }
Hochschule&Hochschule für Technik - FHNW\\
Studiengang&Elektro- und Informationstechnik\\
Auftraggeber&Dr. Luca Dalessandro\\
Betreuer&Prof. Dr. Sebastian Gaulocher \newline Prof. Peter Niklaus \newline Prof. Dr. Richard Gut \newline  Dr. Anita Gertiser \newline Pascal Buchschacher \\
Autoren&\textbf{Gruppe 1} \newline Niklaus Schwegler \newline Lukas von Däniken \newline Pascal Puschmann  \newline Simon Rohrer \newline Marco Binder\\
Version&2.0 %Normally not used!
\end{tabular}
\end{center}
}

\clearpage

%\selectlanguage{ngerman}				%ngerman or english
\thispagestyle{empty}
			
\begin{abstract}
Abstract

\end{abstract}
Die Firma Schaffner entwickelt EMI-Filter. Um diese zu modellieren, benötigen sie eine Simulationssoftware. Die Hauptanwendung dieser Software besteht darin, die Einfügedämpfung in Abhängigkeit der parasitären Parameter der Bauteile,  über ein Frequenzspektrum bis 30Mhz zu visualisieren. Die Software sollte plattformunabhängig und modular erweiterbar sein. Aus diesem Grund basiert  das Programm auf der Model-View-Controller-Struktur, geschrieben in Java. Für die Berechnung wurden jeweils vereinfachte Ersatzschaltungen für Gleich- und Gegentaktstörungen modelliert. Die einzelnen Bauteile sind modular in die Software implementiert. Diese einzelnen Bauteile werden nun anhand der Ersatzschaltung aneinander gereiht, wodurch diese berechnet wird. Auf der Benutzeroberfläche sind für alle verstellbaren Parameter Slider erzeugt, um die Werte um +/-30\% zu verändern. Dabei werden die Auswirkungen für beide Signale direkt in zwei einzelnen Plots sichtbar. Die Software ist somit in der Lage, Filter direkt miteinander zu vergleichen. Sie kann zudem einfach auf andere Schaltbilder erweitert werden. 
%%---ABSTRACT----------------------------------------------------------------------------
\selectlanguage{ngerman}                %ngerman or english
\thispagestyle{empty}
\begin{abstract}
Abstract
nanananananan baat maaaan
\end{abstract}




%%---TABLE OF CONTENTS-------------------------------------------------------------------
\pagenumbering{Roman}
\selectlanguage{ngerman}                %ngerman or english
\tableofcontents
\clearpage
%%---TEXT--------------------------------------------------------------------------------
\pagenumbering{arabic}

\section{Einleitung}
\subsection{Ausgangslage}

Die Firma Schaffer ist der Auftraggeber des Projekt 2 des Studiengangs Elektro- und Informationstechnik. Der Auftrag ist eine GUI zu entwickeln, die das Verhalten von EMI- Filter simuliert. 
 
Unser Team wird durch Niklaus Schwegler geleitet. 
Bei wöchentlichen Sitzungen, bei denen alle Teammitglieder teilnehmen, wird sich auf den neusten Stand gebracht. Das Team wird grob in 3 Gruppen eingeteilt: Elektrotechnik, Software und Organisation. Die Kommunikation innerhalb des Teams wird über einen Discordserver realisiert.  Als Sammelpunkt für alle Dateien, die innerhalb des Projektes erstellt werden, wird Github verwendet. So ist es für alle ersichtlich wer welches Dokument geschrieben hat und was angepasst wurde. Um Diese zu erstellen wird LaTeX verwendet. 

    








\section{Grundlagen} \label{sec:grundlagen}

\subsection{Elektrotechnik} \label{subsec:elektrotechnik}

\subsubsection{Aufbau eines EMI-Filters} \label{subsubsec:emi_filter}
Ein EMI-Filter ist ein lineares Netzwerk aus R, L, C Gliedern und einem Transformator. Somit besitzen sie eine reziproke Übertragungssymetrie, was eine einfache Berechnung von verschiedenen Zusammenhängen erlaubt. 



\begin{figure}[H]
	\centering
	\includegraphics[width = 10cm]{orig_ElectricalCircuit.png}
	\caption{Original Schaltung \cite{aufgabenstellung}}
	\label{fig:orig_Schaltung}
\end{figure}

\subsubsection{Parasitäre Paramter}\label{subsec:parasitparam}
In diesem Unterkapitel werden grundsätzlich die Einflüsse und Eigenschaften von Parasitären Paramentern in Realen Bauteilen, besonders Spule und Kondensator, erklärt

Hierbei müssen die elektrischen Bauelemente, wie Spule und Kondensator mit den passenden parasitären Parameter ergänz werden. In Abbildung \ref{fig:stray_L} und \ref{fig:stray_C} werden die parasitären Parameter von Spule und Kondensator gezeigt.
\begin{figure}[H]
	\begin{minipage}[h]{0.45\linewidth}
		\centering
		\includegraphics[width = 5cm]{stray_L.png}
		\label{fig:stray_L}
		\caption{Parasiäre Elemente einer Induktivität \cite{aufgabenstellung}}
	\end{minipage}
	\begin{minipage}[h]{0.45\linewidth}
		\centering
		\includegraphics[width = 7cm]{stray_C.png}
		\label{fig:stray_C}
		\caption{Parasiäre Elemente einer Kapazität \cite{aufgabenstellung}}
	\end{minipage}
\end{figure}
\subsubsection{Gleich- Gegentaktschaltung} \label{subsubsec:gegentakt}

In der realen Stromverteilung wird beabsichtigt, dass der Stromfluss über einen Zuleiter zum Verbraucher hinein-, respektive über einen Ableiter herausgeführt wird. 
Diese Art der Signalübertragung wird als Gegentakt-Betrieb bezeichnet. Im realen Stromnetz ist allerdings auch der sogenannte Gleichtakt-Betrieb vorhanden. Dabei wirken alle Leiter als Zuleiter, der gesamte Strom wird durch die Erde weggeleitet. Durch das Gesetz der Superposition ist es möglich, den Gleich- und den Gegentaktanteil getrennt voneinander zu betrachten. Dieses Phänomen wird in der Abbildung \ref{fig:auftrennen_der_leitung} dargestellt.  

\begin{figure}[H]
	\centering
	\includegraphics[width=15cm]{grundsatz_cm_dm.jpg}
	\caption{Beispiel aufgetrennten Leitung \textcolor{red}{\textbf{TODO:Quelle bild}}  }
	\label{fig:auftrennen_der_leitung}
\end{figure} 


An einen geerdeten Verbraucher sind 2 Phasen angeschlossen. An der Zuleitung liegt eine Spannung von 15 Volt an, an der Rückleitung liegen -5 Volt an. Diese Leitung wird nun aufgeteilt in eine Gleichtaktleitung, bei welcher über beide Phasen 5 Volt eingespeist werden und in eine Gegentaktleitung, in welcher durch die Zuleitung 10 Volt, respektive in der Rückleitung -10V eingespiesen werden. Während in der Gleichtaktleitung die addierten 10 Volt gegenüber der Erde anliegen, werden sie in der Gegentaktleitung abgeführt. \textcolor{red}{\textbf{TODO:Quelle ganzes kapitel}}


\newpage





\subsubsection{Einfügedämpfung}\label{subsec:einfuge}


Einfügungsdämpfung (engl. Insertion loss) wird in Dezibel (dB) angegeben. Im folgendem wird beschrieben, wie die Einfügedämpfung für die Software errechnet wird.  Die Einfügungsdämpfung werden analytisch ermittelt. Im ersten Schritt werden die Berechnungen in MATLAB getätigt und geplotet. Diese Plots werden dann mit Simulationen in MPLAB Mindi verglichen um festzustellen ob diese korrekt sind. Nach überprüfung auf vollständikeit und korrektheit der Berechnungen, können diese in das Java implementiert werden.  Um die Einfügungsdämpfung bestimmen zu können, wird das Model der 2-Tore verwendet. Einzelne Schaltungsteile werden in der Kettenmatrix \ref{Kettenmatrix} abgebildet, welche dann durch Kaskadierung der einzelnen Kettenmatrixen zusammengeführt werden. Die Einfügungsdämpfung werden aus den Streuparameter\ref{subsec:Streuparameter} abgeleitet, welche direkt aus der Kettenmatrix berechnet werden.\cite{hftech}.



\subsubsection{Streuparameter}\label{subsubsec:streuparameter}
Dieses Kapitel ist eine Kurzeinführung in die Thematik der Streuparameter. Als Grunlage dazu dienen die Quellen \cite{hftech}\cite{Bernstein2015}. Streuparameter (S-Parameter) werden in der Hochfrequenztechnik verwendet, um das Verhalten von n-Toren zu beschreiben. Bei einem 2-Tor sind vier Streuparameter nötig, um das Verhalten zu beschreiben. Sie beschreiben die Transmission von Tor 1 zu Tor 2, sowie von Tor 2 zu Tor 1. Des weiteren zeigen sie die Reflexion an den Toren auf. Die Abbildung \ref{fig:2-Tor} zeigt die Streuparameter an einem 2-Tor. 


\begin{figure}[H]
	\centering
	\includegraphics[width=15cm]{s_params_def.png}
	\caption{2-Tor Wellengrössen und Anschlussleitungen \cite{hftech}}
	\label{fig:2-Tor}
\end{figure}

Bei den S-Parametern werden die Eingangs- und Ausgangsgrössen nicht direkt anhand elektrischer Ströme und Spannungen beschrieben, sondern mithilfe von Wellengrössen, wobei a\textsubscript{i} die einlaufenden Wellen sind und b\textsubscript{i} die reflektierenden Wellen. Der Index i stellt den Torindex dar. Formel \ref{equ:def_a} und \ref{equ:def_b} zeigen wie die Wellengrössen a\textsubscript{i} sowie b\textsubscript{i} definiert sind. Die Quadrate der Beträge der Wellenstärken a und b entsprechen gerade den Leistungen, die mit diesen Wellen übertragen werden.

\begin{equation}\label{equ:def_a}
	a_{ i } = \sqrt{ P_{ vor } }, a = Wellenstärke\;der\;vorlaufenden\;Welle
\end{equation}
\begin{equation}\label{equ:def_b}
	b_{ i } = \sqrt{ P_{ rück } }, b = Wellenstärke\;der\;rücklaufenden\;Welle
\end{equation}

Aus der Abbildung 2.3 lässt sich folgende Streumatrix darstellen (Formel \ref{equ:scatteringMatrix}):
\begin{equation}\label{equ:scatteringMatrix}
	\left[
		\begin{matrix}b_1 \\ b_2 \end{matrix}
	\right]
 	=
 	\left[
 		\begin{matrix}
			s_{11}&s_{12} \\s_{21}&s_{22}
		\end{matrix}
	\right]
	\cdot 
	\left[
		\begin{matrix}
			a_1\\a_2
		\end{matrix}
	\right]
\end{equation}

Die Elemente der S-Matrix sind:

\begin{equation}\label{equ:def_s11}
	s_{11} = b_1/a_1\textsf{ Eingangsreflexionsfaktor bei angepasstem Ausgang (a\textsubscript{2}=0)}
\end{equation}
\begin{equation}\label{equ:def_s12}
	s_{12} = b_1/a_2\textsf{ Rückwärtstransmissionsfaktor bei angepasstem Eingang (a\textsubscript{1}=0)}
\end{equation}
\begin{equation}\label{equ:def_s21}
	s_{21} = b_2/a_1\textsf{ Vorwärtstransmissionsfaktor bei angepasstem Ausgang (a\textsubscript{2}=0)}
\end{equation}
\begin{equation}\label{equ:def_s22}
	s_{22} = b_2/a_2\textsf{ Ausgangsreflexionsfaktor bei angepasstem Eingang (a\textsubscript{1}=0)}
\end{equation}

In der folgenden Schaltung (Abbildung \ref{fig:bspscattering}) wird Schritt für Schritt erklärt, wie der \\ $s_{21}$-Parameter der Streumatrix berechnet wird.
 
\begin{figure}[H]
	\centering
	\includegraphics[width=15cm]{bspscattering.png}
	\caption{Beispielschaltung Streuparameter}
	\label{fig:bspscattering}
\end{figure}

Der Streuparameter $s_{21}$ ist, wie in Formel \ref{equ:s21} definiert. Somit muss im ersten Schritt die Kettenmatrix der Gesamtschaltung ermittelt werden. Die Kettenmatrix bezieht sich auf die beiden Widerstände $R_1$ und $R_2$. Der Widerstand $R_1$ stellt eine Längsimpedanz dar, welche somit in die passende Matrix eingesetzt wird (Formel \ref{equ:horizImpedance}). $A_1$ (Formel \ref{equ:r1}) stellt die Kettenmatrix der Längsimpedanz $R_1$ dar.
\begin{equation}\label{equ:r1}
	A_1 = \left[\begin{matrix}
			1&R_1\\0&1
			\end{matrix}\right]
\end{equation}

Der Widerstand $R_2$ stellt eine Querimpedanz dar, welche in die Formel \ref{equ:verticImpedance} eingesetzt wird. $A_2$ (Formel \ref{equ:r2}) stellt die Kettenmatrix der Querimpedanz $R_2$ dar.
\begin{equation}\label{equ:r2}
			A_2 = \left[\begin{matrix}
			1&0\\\frac{1}{R_2}&1
			\end{matrix}\right]
\end{equation}

Durch Kaskadieren der beiden Kettenmatrizen wird die Kettenmatrix der Gesamtschaltung gebildet. Kaskadieren bedeutet das Bilden des Matrixproduktes.

\begin{equation}\label{equ:bspSMatrixtot}
A_{tot} = A_1 \cdot A_2 = \left[\begin{matrix}
			1&R_1\\0&1
			\end{matrix}\right] \cdot  \left[\begin{matrix}
			1&0\\\frac{1}{R_2}&1
			\end{matrix}\right]
			= \left[\begin{matrix}
			1&20\\0&1
			\end{matrix}\right] \cdot \left[\begin{matrix}
			1&0\\\frac{1}{150}&1
			\end{matrix}\right] =
			\left[\begin{matrix}
			1.1333&20.000\\0.0067&1.0000
			\end{matrix}\right]
\end{equation}

Der nächste Schritt besteht darin, die Kettenmatrix in Bezug zu den Wellenimpedanzen darzustellen. In diesem Beispiel entspricht die Wellenimpedanz des Innenwiderstandes sowie des Lastwiderstandes 50$\Omega$ . Aus der Formel \ref{equ:bspSMatrixtotnorm}  wird diese durch Einsetzen gebildet.

\begin{equation}\label{equ:bspSMatrixtotnorm}
			A'_{tot} = \left[\begin{matrix}
			A_{11}\cdot\sqrt{ \frac{ R_{ l } }{ R_{ q } } }
			&
			\frac{A_{12}}{\sqrt{R_q \cdot R_l}}
			\\
			A_{21} \cdot \sqrt{R_q \cdot R_l}
			&
			A_{22} \cdot \sqrt{\frac{R_q}{R_l}}
			\end{matrix}\right]
			=
			\left[\begin{matrix}
			1.1333\cdot\sqrt{ \frac{ 50}{50} }
			&
			\frac{20.000}{\sqrt{50\cdot 50}}
			\\
			0.0067 \cdot \sqrt{50 \cdot 50}
			&
			1.0000 \cdot \sqrt{\frac{50}{50}}
			\end{matrix}\right]
			=
			\left[\begin{matrix}
			1.1333
			&
			0.4000
			\\
			0.3333
			&
			1.0000
			\end{matrix}\right]
\end{equation}
Aus der Matrix $A'_{tot}$ kann mithilfe der Formel \ref{eqn:s21ausA} der Streuparameter $s_{21}$ direkt ausgerechnet werden.

\begin{equation}\label{eqn:s21ausA}
	s_{21} = \frac{2}{A'_{11}+A'{12}+A'_{21}+A'_{22}}
	=
	\frac{2}{1.1333+0.4000+0.3333+1.0000}
	= 0.6977
\end{equation}
Der Streuparameter $s_{21}$ entspricht dem Transmissionsfaktor der eingehenden Welle. Daraus ergibt sich die Formel \ref{eqn:s32ausLeistung}. Dies dient der Überprüfung der vorigen Berechnungen. Die Resultate der beiden Berechnungen stimmen überein.

\begin{equation} \label{eqn:s32ausLeistung}
	s_{21} = \sqrt{ \frac {P_l}{P_{AV}}} = \sqrt{ \frac{0.2434W} {0.5000W}} = 0.6977
\end{equation}

\bigskip
\subsubsection{Kettenmatrix}\label{subsec:kettenmatrix}

Die ABCD-Matrix ist eine weitere gängige Variante, um das Verhalten von 2-Toren zu beschreiben. Diese Variante hat den Vorteil, dass man in Serie geschaltene 2-Tore ohne grossen Aufwand zusammen rechnen kann. Sobald man die einzelnen ABCD-Matrixen gebildet hat und die Schaltung soweit vereinfacht ist, dass nur noch in Serie geschaltene ABCD-Matrixen vorzufinden sind, können diese miteinander multipliziert werden. Das Matrix-Produkt stellt dann die ABCD-Matrix der Gesamtschaltung dar. Folgende gängigen Schaltungen helfen die ABCD-Matrixen der einzelnen Schaltungsteilen zu bilden.

Die Längsimpedanz lässt sich anhand der ABCD-Matrix A\textsubscript{L} (Formel \ref{equ:horizImpedance}) darstellen
\begin{figure}[H]
	\begin{minipage}[h]{0.45\linewidth}
		\centering
		\includegraphics[width = 3cm]{h_impedance.png}
		\caption{Längsimpedanz \cite{2torTabelle}}
	\end{minipage}
	\begin{minipage}[h]{0.45\linewidth}
		\centering
		\begin{equation}\label{equ:horizImpedance}
			A_L = \left[\begin{matrix}
			1&\underline{Z}_b\\0&1
			\end{matrix}\right]
		\end{equation}
	\end{minipage}
\end{figure}
Die Querimpedanz lässt sich anhand der ABCD-Matrix A\textsubscript{Q} (Formel \ref{equ:verticImpedance}) darstellen
\begin{figure}[H]
	\begin{minipage}[h]{0.45\linewidth}
		\centering
		\includegraphics[width = 3cm]{v_impedance.png}
		\caption{Querimpedanz \cite{2torTabelle}}
	\end{minipage}
	\begin{minipage}[h]{0.45\linewidth}
		\centering
		\begin{equation}\label{equ:verticImpedance}
			A_Q = \left[\begin{matrix}
			1&0\\\frac{1}{\underline{Z}_a}&1
			\end{matrix}\right]
		\end{equation}
	\end{minipage}
\end{figure}
Die Impedanz eines T-Glieds lässt sich anhand der ABCD-Matrix $A_T$ (Formel \ref{equ:tImpedance}) darstellen
\begin{figure}[H]
	\begin{minipage}[h]{0.45\linewidth}
		\centering
		\includegraphics[width = 3cm]{t_impedance.png}\label{fig:tImpedance}
		\caption{T-Glied \cite{2torTabelle}}
	\end{minipage}
	\begin{minipage}[h]{0.45\linewidth}
		\centering
		\begin{equation}\label{equ:tImpedance}
			A_T = \left[\begin{matrix}
			1+\frac{\underline{Z}_2}{\underline{Z}_3}&\underline{Z}_1+\underline{Z}_3+\frac{\underline{Z}_1\underline{Z}_3}{\underline{Z}_2}\\
			\frac{1}{\underline{Z}_2}&1+\frac{\underline{Z}_3}{\underline{Z}_2}
			\end{matrix}\right]
		\end{equation}
	\end{minipage}
\end{figure}
Die Impedanz eines  $\pi$-Glieds lässt sich anhand der ABCD-Matrix $A_\pi$ (Formel \ref{equ:piImpedance}) darstellen
\begin{figure}[H]
	\begin{minipage}[h]{0.45\linewidth}
		\centering
		\includegraphics[width = 3cm]{pi_impedance.png}
		\label{fig:piImpedance}
		\caption{Pi-Glied \cite{2torTabelle}}
	\end{minipage}
	\begin{minipage}[h]{0.45\linewidth}
		\begin{equation}\label{equ:piImpedance}
			A_\pi = \left[\begin{matrix}
			1+\frac{\underline{Z}_2}{\underline{Z}_3}&\underline{Z}_2\\
			\frac{1}{\underline{Z}_1}+\frac{1}{\underline{Z}_3}+\frac{\underline{Z}_2}{\underline{Z}_1\underline{Z}_3}&1+\frac{\underline{Z}_2}				{\underline{Z}_1}
			\end{matrix}\right]
		\end{equation}
	\end{minipage}
\end{figure}

Wenn die ABCD-Matrix einer Schaltung gebildet wurde, kann diese direkt in die Streuparameter umgewandelt werden. Der $s_{21}$ Parameter kann wie in Formel \ref{equ:s21} beschrieben, durch einsetzten der ABCD-Matrix bestimmt werden. Für den Widerstand $R_w$ muss die verwendete Bezugsimpedanz eingesetzt werden.
\begin{equation}\label{equ:s21}
s_{21} = \frac{2}{A_{11}+\frac{A{12}}{R_w}+A_{21}*R_w+A_{22}}
\end{equation}
Die Indexierung der ABCD-Matrix wird in Abbildung \ref{equ:A_index} gezeigt
\begin{equation}\label{equ:A_index}
	A = \left[\begin{matrix}
	A_{11}&A_{12}\\A_{21}&A_{22}
	\end{matrix}\right]
\end{equation}
\newpage


\subsection{Programmieren} \label{subsec:softech}
\subsubsection{MVC-Struktur}\label{subsec:mvc}

Das MVC-Framework wird zur Softwarestrukturierung verwendet. Durch diese Strukturierung werden die Berechnungen der Daten (eng. model), die Steuerung (engl. controller) und dessen graphischer Repräsentation (engl. view) getrennt. In der Abbildung \ref{fig:MVCBeispiel} ist dieser Aufbau in einem Beispielklassendiagramm dargestellt. \cite{MVCDesignPattern}

\begin{figure}[H]
	\centering
	\includegraphics[width = 10cm]{MVC_Beispiel.png}
	\caption{MVC Beispielklassendiagramm \cite{MVCBeispiel}}
	\label{fig:MVCBeispiel}
\end{figure}

Der Ablauf dieser Struktur ist wie folgt: 

\begin{enumerate}
\item Benutzereingabe löst Event aus
\item Die Aktion wird dem Controller übergeben. Dieser holt die Daten in der View, leitet diese dem Model weiter und löst die Berechnungen aus
\item Das Model führt die Berechnungen aus und informiert den Observer
\item Der Observer löst ein Event in der View aus. View kann die Daten vom Model holen und ausgeben
\end{enumerate}




\section{Software} \label{sec:software}

Die Software ist als Model-View-Controller Entwurf aufgebaut. Diese Struktur wird im Anhang erläutert. Somit sind die Berechnungen von der Benutzeroberfläche getrennt. Während im Model die Schaltbilder aufgebrochen werden um die Einfügedämpfung zu berechnen wird die Benutzeroberfläche in einzelne Panels aufgeteilt. Diese Panels beinhalten jeweils eine Grundfunktion der Software. Im Klassendiagramm (Abbildung \ref{fig:klassendiagramm} ) sind die Inhalte der verschiedenen Programmteile, sowie deren zusammenspiel ersichtlich. Der Controller wird standardmässig nur zur Übergabe der Befehle zwischen View und Model verwendet. Ausserhalb dieser Struktur befindet sich noch das Framework, welches die gesamte Software intialisiert. Im Klassendiagramm sind auch noch weitere externe Klassen vorhanden, welche der Software verschiedene Methoden zur Verfügung stellen. Diese gehören nicht zum MVC Entwurf da sie allgemeingültig sind und nicht für ein spezifisches Problem erstellt wurden.


\begin{figure}[H]
		\centering
		\includegraphics[width = 16cm]{Klassendiagramm.png}
		\label{fig:klassendiagramm}
		\caption{Klassendiagramm}
\end{figure}
\newpage

\subsection{Ermittlung der Einfügedämpfung} \label{subsec:ermittlung}
Aufbrechen der Schaltung, beschrieb der Klassen und Methoden des Models

\begin{figure}[H]
		\centering
		\includegraphics[width = 6cm]{EMI_CM.png}
		\label{fig:cmschaltung}
		\caption{Gleichtakt}
\end{figure}

\begin{figure}[H]
		\centering
		\includegraphics[width = 6cm]{EMI_DM.png}
		\label{fig:dmschaltung}
		\caption{Gegentakt}
\end{figure}



\subsection{Benutzeroberfläche} \label{subsec:benutzeroberflaeche}

\begin{figure}[H]
		\centering
		\includegraphics[width = 15cm]{GUI_unfertig.jpg}
		\label{fig:gui}
		\caption{Benutzeroberflächre unfertig}
\end{figure}


Aufbau unserer GUI zeigen, Datenverarbeitung aufzeigen und Zusammenspiel der Panels erklären

Danach subsub's der einzelnen Panels um deren Funktionen zu zeigen


\subsubsection{Plotpanel} \label{subsubsec:plotpanel}



\subsubsection{Menu}\label{subsubsec:menu}


\subsubsection{Inputpanel} \label{subsubsec:inputpanel}


\subsubsection{Filtertabelle} \label{subsubsec:filtertabelle}






\section{Aufbau der Software} \label{sec:software}

\subsection{Klassendiagramm} \label{subsec:klassendiagramm}

\subsection{Ersatzschaltbilder} \label{subsec:ersatzschaltbilder}
Aufteilung in Teilprobleme und einzelne Methoden
\subsection{GUI} \label{subsec:gui}
\subsubsection{Menu}\label{menu}

\subsection{Datenverarbeitung} \label{subsec:datenverarbeitung}

\subsection{Datenpräsentation} \label{subsec:datenpräsentation}

\subsection{Speicherverwaltung} \label{subsec:speicherverwaltung}

Programmablauf beschreiben?




\section{Testkonzept}\label{sec:testkonzept}
Wieso, weshalb und Warum
\subsection{Prinzip} \label{prinzip}
Wie ist das Testkonzept aufgebaut
\subsection{Validierung} \label{validierung}
Testergebnisse darstellen und Interpretieren

Ebenfalls wird hier beschrieben welche Werte wir mit der Simulation und mit Matlab erreicht haben

\section{Schluss} \label{sec:schluss}
Zum Abschluss des Projektes und nachdem alle Test durchgeführt wurden, wird die Software dem Auftraggeber zur Verfügung gestellt. Es konnten alle muss-Ziele , die im Pflichtenheft gesteckt wurden zufriedenstellend umgesetzt werden. Die Software ist in der Lage die Eifügedämpfung eines EMI-Filters grafisch darzustellen. Ausserdem ist es möglich verschiedene Filterprofile anzulegen und sie zu speichern. Die verschieden Filter können gleichzeitig im Plot angezeigt werden. 

Die grösste Herausforderung während des Projektes bestand aus der Erarbeitung der elektrotechnischen Grundlagen und die Berechnung der Eifügedämpfung des Netzwerkes in DM und CM. Die Entwicklung des Softwaregrundgerüsts ging zügig vorwärts. Die Schwierigkeiten bei der Software waren Beispielsweise die Implementierung der durchgeführten Berechnungen in den Code und die Verarbeitung der Werte, die in die GUI eingegeben werden. Um die Performence nicht einzuschränken können nur 10 Filterprofile gleichzeitig angezeigt werden.
Schlussendlich konnten alle Schwierigkeiten gut gelöst werden. Die Software könnte natürlich noch weiterentwickelt werden.  Eine weitere sinnvolle Funktion wäre eine Monte-Carlo Analyse. Diese wurde jedoch, aus Zeitgründen nicht implementiert. Als Abschliessendes Fazit ist zu sagen, dass dieses Projekt erfolgreich und Termingerecht durchgeführt werden konnte. 



%%---BIBLIOGRAPHY------------------------------------------------------------------------

{\sloppypar
\selectlanguage{ngerman}	
\setlength{\bibitemsep}{\baselineskip}
\printbibliography[heading=bibintoc]
\label{sec:lit}
}

%%---Anhang------------------------------------------------------------------------

\section{Anhang} \label{sec:anhang}


\subsection{Testkonzept} \label{subsec:eltech}
\begin{figure}[H]
	\centering
	\includegraphics[width=16cm]{Protokoll.png}
	\label{fig:Protokoll}
\end{figure}

\begin{figure}[H]
	\centering
	\includegraphics[width=16cm]{uebersicht.png}
	\label{fig:übersicht}
\end{figure}

\begin{figure}[H]
	\centering
	\includegraphics[width=16cm]{Test1.png}
	\label{fig:Test1}
\end{figure}

\begin{figure}[H]
	\centering
	\includegraphics[width=16cm]{Test2.png}
	\label{fig:Test2}
\end{figure}

\begin{figure}[H]
	\centering
	\includegraphics[width=16cm]{Test3.png}
	\label{fig:Test3}
\end{figure}

\begin{figure}[H]
	\centering
	\includegraphics[width=16cm]{Test4.png}
	\label{fig:Test4}
\end{figure}

\begin{figure}[H]
	\centering
	\includegraphics[width=16cm]{Test5.png}
	\label{fig:Test5}
\end{figure}

\begin{figure}[H]
	\centering
	\includegraphics[width=16cm]{Test6.png}
	\label{fig:Test6}
\end{figure}

\begin{figure}[H]
	\centering
	\includegraphics[width=16cm]{Test7.png}
	\label{fig:Test7}
\end{figure}

\begin{figure}[H]
	\centering
	\includegraphics[width=15cm]{Abnahme1.png}
	\label{fig:Protokoll}
\end{figure}
\begin{figure}[H]
	\centering
	\includegraphics[width=15cm]{Abnahme2.png}
	\label{fig:Protokoll}
\end{figure}
\begin{figure}[H]
	\centering
	\includegraphics[width=15cm]{Abnahme3.png}
	\label{fig:Protokoll}
\end{figure}



%%---NOTES for DEBUG---------------------------------------------------------------------
\ifdraft{%Do this only if mode=draft
%%requires \usepackage{todonotes})
\newpage
\listoftodos[\section{Todo-Notes}]
\clearpage
}
{%Do this only if mode=final
}
\end{document}
