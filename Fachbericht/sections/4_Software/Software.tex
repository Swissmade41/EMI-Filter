
\section{Software} \label{sec:software}
In diesem Kapitel wird die Umsetzung der Berechnung (Kapitel \ref{sec:umsetzung}) und der anderen Anforderungen an die Software in Java dokumentiert. Die programmiertechnischen Grundlagen sind in Kapitel \ref{sec:grundlagen} beschrieben.
\subsection{Übersicht} \label{subsec:uebersicht}

Das Programm simuliert die Einfügedämpfung eines EMI-Filters. Solche Filter werden häufig in Schaltnetzteile verbaut, um zu verhindern, dass Störungen zurück ins Netz gespeist werden.  Die Benutzeroberfläche des Programms ist in der Abbildung \ref{fig:GUI} ersichtlich.

\begin{figure}[H]
	\centering
	\includegraphics[width=13cm]{gui.png}
	\caption{GUI}
	\label{fig:GUI}
\end{figure} 

Die Software ist nach dem Model-View/Controller Entwurfsmuster \cite{MVCDesignPattern} aufgebaut. Das Klassendiagramm der Software befindet sich im Anhang (TODO: Verweis).  Die View ist in fünf Panels unterteilt. Das InputPanel (1) beinhaltet für jeden Filterparamter ein Textfeld, einen Slider und ein Anzeigelabel, welche die Parameterwerte definieren. Das  FiltertablePanel (2) verwaltet die verschiedene Filter in einer Tabellen, in der auch die Parameterwerte hinterlegt werden. Das ButtonPanel (3) besitzt vier Buttons, die für die Speicher- und Filterverwaltung zuständig sind. Die Menubar (4) besitzt verschiedene Menüs für die Bedienung des Programms. Das PlotPanel (5) dient zur grafischen Darstellung der Einfügeverluste des Filters. Der Controller leitet Informationen von der View zu dem Model weiter. Das Model besitzt Klassen, um mit die Komponente der modellierten realen Bauteile zu rechnen. Via eines Observer wird ein Update in der View ausgelöst.

\newpage

%Diese Infos gehören auch noch rein
%Die Oben abgebildete Benutzerfläche ist die Maske, welche sich beim aufstauten der Software auf dem Bildschirm präsentiert. Wie die  meisten gängigen Applikationen verfügt die GUI eine am oberen Rand platzierten Menubar, um Dateien zu verwalten sowie Hilfestellung zu bieten. Im Zentrum ist im oberen Bereich das Plotpanel platziert, welches die Simulation sowohl im Gleichtakt (CM), als auch im Gegentakt (DM) als grafische Kurve darstellt. Gleich unterhalb befindet sich das Inputpanal, es ermöglicht mittels den Textfeldern eine  effiziente Eingabe der  Parameter. Mittels den Sleidern können die Parameter prozentual angepasst werden. Die Filtertabelle links dient zur Verwaltung der Filterprofilen.



\subsection{View} \label{subsec:view}
Die View ist als GridBagLayout organisiert und enthält die im Kapitel \ref{subsec:uebersicht} erwähnten Panels.
\bigskip

\paragraph{Inputpanel} \label{par:inputpanel}
Das InputPanel hat den Controller und ist im GridBagLayout organisiert. Das Panel wird in mehreren Subpanels unterteilt. Diese sind in der Abbildung (TODO: verweis) abgebildet. 

%TODO Bild Inputpanel Bearbeitet

Das InformationPanel(1) ist im Null-Layout organisiert. Es werden die Prozentzahlen der Schieberegler dargestellt. Diese sind an einer festen Position, da die Grösse des Inputpanels nicht verändert werden kann. Die weiteren Subpanels (2-5) sind im GridBagLayout organisiert. In diesen Panels befinden sich die Komponenten, der modellierten realen Bauteile. Jede Komponente besitzt ein Textfeld, einen Slider und ein Label. Im Textfeld kann der Benutzer seine Werte für die Komponente eintragen. Beim Aufstarten des Programmes wird das Feld mit einem Standardwert, der vom Auftragdokument übernommen wurde, geladen. Mit der Klasse JEngineerField werden die Eingaben geprüft. Es ist nur möglich Zahlen einzugeben. Grosse und kleine Zahlen können zur Vereinfachung in wissenschaftlicher Schreibweise (18e-12) oder in Einheiten-Schreibweise (18p) eingetragen werden. Die Ausgabe ist als Einheiten-Schreibweise vordefiniert. Mit einem Rechtsklick auf das Textfeld ist es möglich die Schreibweise der Eingabe und Ausgabe individuell zu verändern.
Mit dem Schieberegler wird der im Textfeld eingegebener Wert um $\pm$ 30\% verändert und im Label als effektiver Wert ausgegeben. Das Textfeld besitzt einen ActionListener und der Slider einen changeListener. Bei einem Ereignis wird die dazugehörige Methode im Controller  aufgerufen.
\bigskip

\paragraph{Filtertablepanel} \label{par:filterpanel}
Das Filtertablepanel hat den Controller und ist im GridBagLayout organisiert. Es enthält eine Tabelle, in der die Parameterwerte hinterlegt und zu dem entsprechenden Filter zugeordnet werden. Jeder Filter in der Tabelle besitzt eine Checkbox, um den Filter zu aktivieren, und ein Textfeld, um den Filter  zu benennen. Die Tabelle hat einen TableModelListener und einen ListSelectionListener. Bei einem Ereignis wird die dazugehörige Methode im Controller  aufgerufen.
\bigskip

\paragraph{Buttonpanel} \label{par:buttonpanel}
Das ButtonPanel hat den Controller und ist im Null-Layout organisiert. Es enthält die vier Buttons und zwei dazugehörige Labels. Mit diesen kann die Filtertabelle verwaltet und die Speicherverwaltung aufgerufen werden. Jeder Button hat einen ActionListener. Bei einem Ereignis wird die dazugehörige Methode im Controller  aufgerufen.
\bigskip

\paragraph{Plotpanel} \label{par:plotpanel}
Das PlotPanel ist im BorderLayout organisiert. Mit Hilfe des Package JfreeChart (TODO: verweis) wird die Einfügedämpfung des EMI-Filters in einem Plot dargestellt. Das Package übernimmt viele Funktionen wie das Zoomen des Plots oder das Ändern der Darstellung. Löst der Observer ein Erreignis aus, werden die Daten vom Model geholt und geplottet.
\bigskip

\newpage

\paragraph{Menubar} \label{par:menu}
Das Menubar hat den Controller und ist im Null-Layout organisiert. Es besitzt die Menus File, in dem die Speicherverwaltung aufgerufen werden kann, Circuit, in dem die Schaltungen, von denen die Berechnungen ausgehen angezeigt werden und About, in dem Informationen über das Programm hinterlegt sind. Die Menüs haben einen ActionListener. Bei einem Ereignis wird die dazugehörige Methode im Controller  aufgerufen.


\subsection{Controller} \label{subsec:controller}

Der Controller leitet die entsprechenden Aufgaben an das Model weiter. Es umfasst neben dem Konstruktor 14 Methoden, die entweder Daten direkt an das Model weiterleiten, um die Berechnungen durchführen zu können, oder Methode wie getEffectiveParameterValues(), die als Schnittstelle dienen, um Daten vom Model in den StorageManager zu verschieben, damit diese abgespeichert werden können.



\subsection{Model} \label{subsec:Model}

Die Hauptaufgabe des Models besteht darin, die Einfügungsdämpfung zu berechnen. Die Methode calculate() berechnet anhand der übergebenen Komponenten der Schaltung die Einfügungsdämpfung. Die Struktur des Models ist dem Klassendiagram im Anhang zu entnehmen. Logarithmisch verteilt werden 400 Werte in einem Bereich von 0Hz bis 30MHz für die beiden Schaltungen (siehe Kapitel \ref{sec:umsetzung}) berechnet. Die Berechnungen werden seperat für Gegen- und Gleichtakt durchgeführt.

Dafür werden, wie in Kapitel \ref{subsubsec:kettenmatrix} beschrieben, die einzelnen Schaltungsteile anhand von Kettenmatrizen beschrieben. Da die Kettenmatrix die Impedanz der einzelnen Schaltungsteilen beinhaltet, gibt es Klassen für Spule, Kondensator und Widerstand (Induktor, Capacity, Resistor). Dem Konstruktor dieser Klassen wird der Hauptwert(beim Kondensator z.B. die Kapazität), sowie die Parasitären Parameter (beim Kondensator der Serie Widerstand und die Serie Kapazität) übergeben. Diese Objekte stellen die Methode .getImpedance(frequency) zu Verfügung. Sie gibt für eine Frequenz die Impedanz zurück.

Die Klasse Mikromatlab dient als Bibliothek für die Berechnungen. Sie beinhaltet die Methoden getSeriesImpedanceMatrix (SeriesImpedance) und getShuntImpedanceMatrix (ShuntImpedance). Diese Methoden geben für eine Impedanz(Quer oder Längs) die entsprechende Kettenmatrix zurück (Kapitel \ref{subsubsec:kettenmatrix}).

Die Methode cascade(1. Matrix, 2. Matrix) ist eine weitere Methode der Mikromatlab Klasse. Sie ermöglicht die Kaskadierung der einzelnen Schaltungsteile (Kettenmatrizen). Sie führt eine Matrix Multiplikation der übergebenen Matrizen durch. Die einzelnen Schaltungsteile werden miteinander multipliziert. Das Produkt stellt die Gesamtkettenmatrix der Schaltung dar. 

Die Formel \ref{equ:s21} legt dar, wie aus der Gesamtkettenmatrix der Streuparameter s21 berechnet wird. In der Methode calculate wird die beschriebene Berechnung in einer for-Schleife für die logarithmisch eingeteilte Frequenz-Achse durchgeführt. Die logarithmische Einteilung wird anhand der Methode logspace() der Klasse Mikromatlab erstellt. Die berechneten s21-Parameter werden logarithmisch umgerechnet und in den Attrributen cmData und dmData abgespeichert. Diese beiden Attribute sind drei dimensionale Arrays, wobei die erste Dimension die Filter der Filtertabelle nummeriert. Die zweite Dimension spezifiziert die x und y-Achse. Die dritte Dimension beinhaltet die errechneten Werte. 

Sobald die Berechnungen gemacht sind, wird anhand der Methode notifyObserver() bei der View ein update ausgelöst. Die Methoden getCM und getDM des Models ermöglichen der View, die berechneten Daten zu holen.

Wie im Kapitel \ref{subsec:view} beschrieben, ist es möglich die Filter anhand der Checkboxen sichtbar und unsichtbar zu stellen. Um diese Funktion zu gewährleisten, wird der Methode calculate() ein Kennzeichenung visability mitgegeben. Eine 1 bedeutet sichtbar und eine 0 unsichtbar. Falls die Funktion des entsprechenden Filters unsichtbar sein soll, wird dies in den Datensätzen mit einer - 1 gekennzeichnet. 


\subsection{Trace} \label{trace}
Um den Ablauf innerhalb der Software nachvollziehen zu können wird nachfolgend die Sogenannte "Trace" beim Aufstarten der Software und beim Auslösen eines Ereignisses aufgezeigt. Die "Trace" wird innerhalb der Software mithilfe der Klasse traceV4 in die Konsole ausgegeben.



