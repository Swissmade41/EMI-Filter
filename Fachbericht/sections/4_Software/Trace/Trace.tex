\subsection{Trace} \label{trace}
Um den Ablauf innerhalb der Software nachvollziehen zu können, wird nachfolgend die sogenannte "Trace" beim Aufstarten der Software (Abbildung \ref{fig:tracestart}) und beim Auslösen eines Ereignisses (Abbildung \ref{fig:traceevent}) aufgezeigt. Die "Trace" wird innerhalb der Software mithilfe der Klasse traceV4 in der Konsole ausgegeben.

\paragraph{Trace beim Aufstarten}
 
 \begin{figure}[H]
	\centering
	\includegraphics[width=16cm]{trace_start.png}
	\caption{Trace beim Aufstarten}
	\label{fig:tracestart}
\end{figure} 

\newpage

\paragraph{Trace bei einem Event}

\begin{figure}[H]
	\centering
	\includegraphics[width=16cm]{trace_event.png}
	\caption{Trace beim event}
	\label{fig:traceevent}
\end{figure} 

\newpage