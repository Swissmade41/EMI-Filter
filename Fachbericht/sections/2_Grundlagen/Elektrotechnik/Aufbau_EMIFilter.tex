\subsubsection{Aufbau eines EMI-Filters} \label{subsubsec:emi_filter}
Ein EMI-Filter ist ein lineares Netzwerk aus R, L, C Bauteilen und einem Transformator. Somit besitzten sie eine reziproke Übertragungssymetrie, was eine einfache Berechnungen von verschiedenen Zusammenhängen erlaubt. Einem reziproken Netzwerk ist die Betriebsrichtung gleichgültig.
Die Schaltung \ref{fig:orig_Schaltung} \nameref{fig:orig_Schaltung} zeigt den Filteraufbau, wie er der Aufgabenstellung zu entnehmen ist. Um das Gegentaktrauschen und das Gleichtaktrauschen bestimmen zu können, werden die beiden Schaltungsäquivalente gebildet. 

\begin{figure}[H]
	\centering
	\includegraphics[width = 10cm]{orig_ElectricalCircuit.png}
	\caption{Original Schaltung \cite{aufgabenstellung}}
	\label{fig:orig_Schaltung}
\end{figure}


%TODO Die Schaltungsäquivalenzen müssen aufgeteilt werden und in die jeweiligen subsubsections gekippt werden
%TODO Reziprokzität einfügen
%TODO Allgemein EMI-Filter beschreiben
