\subsubsection{Gegentaktschaltung} \label{subsec:gegentakt}

In der realen Stromverteilung wird beabsichtigt, dass der Stromfluss über einen Zuleiter zum Verbraucher ein-, respektive über einen Ableiter herausgeführt wird. 
Diese Art der Signalübertragung wird als Gegentakt-Betrieb bezeichnet. Im realen Stromnetz ist allerdings auch der sogenannte Gleichtakt-Betrieb vorhanden. Dabei wirken alle Leiter als Zuleiter, der Gesamte Strom wird durch die Erde herausgeführt. Da elektrischer Strom als linear behandelt wird gilt das Gesetz der Superposition. Dadurch ist es möglich, den Gleichtakt- und den Gegentaktanteil getrennt voneinander zu betrachten.
 
 Abbildungen einfügen
 
 Dieses Phänomen wird durch ein Beispiel klar. An einen geerdeten Verbraucher sind 2 Phasen angeschlossen. An der Zuleitung liegt eine Spannung von 15 Volt an, an der Rückleitung liegen -5 Volt an. Diese Leitung wird nun aufgeteilt in eine Gleichtaktleitung, bei welcher über beide Phasen 5 Volt eingespeist werden und in eine Gegentaktleitung, in welcher durch die Zuleitung 10 Volt, respektive in der Ableitung -10V eingespiesen werden. Während in der Gleichtaktleitung die Addierten 10 Volt gegenüber der Erde anliegen, werden sie in der Gegentaktleitung abgeführt.
 
 



 Im Stromnetz entstehen Störungen, 

Da sehr viele Verbraucher gleichzeitig auf das Stromnetz einwirken und durch von Ausserhalb induzierte Magnetfelder entstehen Störungen