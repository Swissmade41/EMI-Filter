\subsubsection{Gleich- Gegentaktschaltung} \label{subsubsec:gegentakt}

In der realen Stromverteilung wird beabsichtigt, dass der Stromfluss über einen Zuleiter zum Verbraucher hinein-, respektive über einen Ableiter herausgeführt wird. 
Diese Art der Signalübertragung wird als Gegentakt-Betrieb bezeichnet. Im realen Stromnetz ist allerdings auch der sogenannte Gleichtakt-Betrieb vorhanden. Dabei wirken alle Leiter als Zuleiter, der gesamte Strom wird durch die Erde weggeleitet. Durch das Gesetz der Superposition ist es möglich, den Gleich- und den Gegentaktanteil getrennt voneinander zu betrachten. Dieses Phänomen wird in der Abbildung \ref{fig:auftrennen_der_leitung} dargestellt.  

\begin{figure}[H]
	\centering
	\includegraphics[width=15cm]{grundsatz_cm_dm.jpg}
	\caption{Beispiel aufgetrennten Leitung \textcolor{red}{\textbf{TODO:Quelle bild}}  }
	\label{fig:auftrennen_der_leitung}
\end{figure} 


An einen geerdeten Verbraucher sind 2 Phasen angeschlossen. An der Zuleitung liegt eine Spannung von 15 Volt an, an der Rückleitung liegen -5 Volt an. Diese Leitung wird nun aufgeteilt in eine Gleichtaktleitung, bei welcher über beide Phasen 5 Volt eingespeist werden und in eine Gegentaktleitung, in welcher durch die Zuleitung 10 Volt, respektive in der Rückleitung -10V eingespiesen werden. Während in der Gleichtaktleitung die addierten 10 Volt gegenüber der Erde anliegen, werden sie in der Gegentaktleitung abgeführt. \textcolor{red}{\textbf{TODO:Quelle ganzes kapitel}}


\newpage




