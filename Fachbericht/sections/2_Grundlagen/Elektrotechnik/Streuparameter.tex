\subsubsection{Streuparameter}\label{subsubsec:streuparameter}
Dieses Kapitel ist eine Kurzeinführung in die Thematik der Streuparameter. Als Grunlage dazu dienen die Quellen \cite{hftech}\cite{Bernstein2015}. Streuparameter (S-Parameter) werden in der Hochfrequenztechnik verwendet, um das Verhalten von n-Toren zu beschreiben. Bei einem 2-Tor sind vier Streuparameter nötig, um das Verhalten zu beschreiben. Sie beschreiben die Transmission von Tor 1 zu Tor 2, sowie von Tor 2 zu Tor 1. Des weiteren zeigen sie die Reflexion an den Toren auf. Die Abbildung \ref{fig:2-Tor} zeigt die Streuparameter an einem 2-Tor. 


\begin{figure}[H]
	\centering
	\includegraphics[width=15cm]{s_params_def.png}
	\caption{2-Tor Wellengrössen und Anschlussleitungen \cite{hftech}}
	\label{fig:2-Tor}
\end{figure}

Bei den S-Parametern werden die Eingangs- und Ausgangsgrössen nicht direkt anhand elektrischer Ströme und Spannungen beschrieben, sondern mithilfe von Wellengrössen, wobei a\textsubscript{i} die einlaufenden Wellen sind und b\textsubscript{i} die reflektierenden Wellen. Der Index i stellt den Torindex dar. Formel \ref{equ:def_a} und \ref{equ:def_b} zeigen wie die Wellengrössen a\textsubscript{i} sowie b\textsubscript{i} definiert sind. Die Quadrate der Beträge der Wellenstärken a und b entsprechen gerade den Leistungen, die mit diesen Wellen übertragen werden.

\begin{equation}\label{equ:def_a}
	a_{ i } = \sqrt{ P_{ vor } }, a = Wellenstärke\;der\;vorlaufenden\;Welle
\end{equation}
\begin{equation}\label{equ:def_b}
	b_{ i } = \sqrt{ P_{ rück } }, b = Wellenstärke\;der\;rücklaufenden\;Welle
\end{equation}

Aus der Abbildung 2.3 lässt sich folgende Streumatrix darstellen (Formel \ref{equ:scatteringMatrix}):
\begin{equation}\label{equ:scatteringMatrix}
	\left[
		\begin{matrix}b_1 \\ b_2 \end{matrix}
	\right]
 	=
 	\left[
 		\begin{matrix}
			s_{11}&s_{12} \\s_{21}&s_{22}
		\end{matrix}
	\right]
	\cdot 
	\left[
		\begin{matrix}
			a_1\\a_2
		\end{matrix}
	\right]
\end{equation}

Die Elemente der S-Matrix sind:

\begin{equation}\label{equ:def_s11}
	s_{11} = b_1/a_1\textsf{ Eingangsreflexionsfaktor bei angepasstem Ausgang (a\textsubscript{2}=0)}
\end{equation}
\begin{equation}\label{equ:def_s12}
	s_{12} = b_1/a_2\textsf{ Rückwärtstransmissionsfaktor bei angepasstem Eingang (a\textsubscript{1}=0)}
\end{equation}
\begin{equation}\label{equ:def_s21}
	s_{21} = b_2/a_1\textsf{ Vorwärtstransmissionsfaktor bei angepasstem Ausgang (a\textsubscript{2}=0)}
\end{equation}
\begin{equation}\label{equ:def_s22}
	s_{22} = b_2/a_2\textsf{ Ausgangsreflexionsfaktor bei angepasstem Eingang (a\textsubscript{1}=0)}
\end{equation}

In der folgenden Schaltung (Abbildung \ref{fig:bspscattering}) wird Schritt für Schritt erklärt, wie der \\ $s_{21}$-Parameter der Streumatrix berechnet wird.
 
\begin{figure}[H]
	\centering
	\includegraphics[width=15cm]{bspscattering.png}
	\caption{Beispielschaltung Streuparameter}
	\label{fig:bspscattering}
\end{figure}

Der Streuparameter $s_{21}$ ist, wie in Formel \ref{equ:s21} definiert. Somit muss im ersten Schritt die Kettenmatrix der Gesamtschaltung ermittelt werden. Die Kettenmatrix bezieht sich auf die beiden Widerstände $R_1$ und $R_2$. Der Widerstand $R_1$ stellt eine Längsimpedanz dar, welche somit in die passende Matrix eingesetzt wird (Formel \ref{equ:horizImpedance}). $A_1$ (Formel \ref{equ:r1}) stellt die Kettenmatrix der Längsimpedanz $R_1$ dar.
\begin{equation}\label{equ:r1}
	A_1 = \left[\begin{matrix}
			1&R_1\\0&1
			\end{matrix}\right]
\end{equation}

Der Widerstand $R_2$ stellt eine Querimpedanz dar, welche in die Formel \ref{equ:verticImpedance} eingesetzt wird. $A_2$ (Formel \ref{equ:r2}) stellt die Kettenmatrix der Querimpedanz $R_2$ dar.
\begin{equation}\label{equ:r2}
			A_2 = \left[\begin{matrix}
			1&0\\\frac{1}{R_2}&1
			\end{matrix}\right]
\end{equation}

Durch Kaskadieren der beiden Kettenmatrizen wird die Kettenmatrix der Gesamtschaltung gebildet. Kaskadieren bedeutet das Bilden des Matrixproduktes.

\begin{equation}\label{equ:bspSMatrixtot}
A_{tot} = A_1 \cdot A_2 = \left[\begin{matrix}
			1&R_1\\0&1
			\end{matrix}\right] \cdot  \left[\begin{matrix}
			1&0\\\frac{1}{R_2}&1
			\end{matrix}\right]
			= \left[\begin{matrix}
			1&20\\0&1
			\end{matrix}\right] \cdot \left[\begin{matrix}
			1&0\\\frac{1}{150}&1
			\end{matrix}\right] =
			\left[\begin{matrix}
			1.1333&20.000\\0.0067&1.0000
			\end{matrix}\right]
\end{equation}

Der nächste Schritt besteht darin, die Kettenmatrix in Bezug zu den Wellenimpedanzen darzustellen. In diesem Beispiel entspricht die Wellenimpedanz des Innenwiderstandes sowie des Lastwiderstandes 50$\Omega$ . Aus der Formel \ref{equ:bspSMatrixtotnorm}  wird diese durch Einsetzen gebildet.

\begin{equation}\label{equ:bspSMatrixtotnorm}
			A'_{tot} = \left[\begin{matrix}
			A_{11}\cdot\sqrt{ \frac{ R_{ l } }{ R_{ q } } }
			&
			\frac{A_{12}}{\sqrt{R_q \cdot R_l}}
			\\
			A_{21} \cdot \sqrt{R_q \cdot R_l}
			&
			A_{22} \cdot \sqrt{\frac{R_q}{R_l}}
			\end{matrix}\right]
			=
			\left[\begin{matrix}
			1.1333\cdot\sqrt{ \frac{ 50}{50} }
			&
			\frac{20.000}{\sqrt{50\cdot 50}}
			\\
			0.0067 \cdot \sqrt{50 \cdot 50}
			&
			1.0000 \cdot \sqrt{\frac{50}{50}}
			\end{matrix}\right]
			=
			\left[\begin{matrix}
			1.1333
			&
			0.4000
			\\
			0.3333
			&
			1.0000
			\end{matrix}\right]
\end{equation}
Aus der Matrix $A'_{tot}$ kann mithilfe der Formel \ref{eqn:s21ausA} der Streuparameter $s_{21}$ direkt ausgerechnet werden.

\begin{equation}\label{eqn:s21ausA}
	s_{21} = \frac{2}{A'_{11}+A'{12}+A'_{21}+A'_{22}}
	=
	\frac{2}{1.1333+0.4000+0.3333+1.0000}
	= 0.6977
\end{equation}
Der Streuparameter $s_{21}$ entspricht dem Transmissionsfaktor der eingehenden Welle. Daraus ergibt sich die Formel \ref{eqn:s32ausLeistung}. Dies dient der Überprüfung der vorigen Berechnungen. Die Resultate der beiden Berechnungen stimmen überein.

\begin{equation} \label{eqn:s32ausLeistung}
	s_{21} = \sqrt{ \frac {P_l}{P_{AV}}} = \sqrt{ \frac{0.2434W} {0.5000W}} = 0.6977
\end{equation}

\bigskip