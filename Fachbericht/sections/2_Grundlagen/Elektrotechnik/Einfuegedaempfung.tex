\subsubsection{Einfügedämpfung}\label{subsec:einfuge}


Die Einfügedämpfung(engl. Insertion loss) ist eine Grösse, die verwendet wird, um das Verhalten einer Schaltung zu beschreiben. Sie beschreibt das Verhältnis zwischen der eingehenden Leistung zur abgegebenen Leistung. Sie wird in Dezibel (dB) angegeben. Es handelt sich um eine logarithmische Grösse. Um die Einfügedämpfung in einem Bereich von bis 30MHz abzudecken wird die Formel \ref{equ:Einfügungsverluste} verwendet. In der Formel \ref{equ:Einfügungsverluste} wird die Einfügedämpfung mittels Streuparameter(S-Parameter) berechnet. Sie wird in Dezibel (dB) angegeben. Der Streuparameter $S_{21}$ beschreibt im wesentlichen den Transmissionsgrad des eingehenden Signals. Der Transmissionsgrad beschreibt, welchen Anteil des eingehenden Signals am Ausgang wieder herauskommt.\\


\begin{equation}\label{equ:Einfügungsverluste}
	IL = -20*log (\left\lvert S_{21} \right\rvert)
\end{equation}

