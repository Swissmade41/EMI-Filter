\section{Einleitung} \label{sec:einleitung}
Entwurf Einleitung P2 EMI-Filter Team 1

Gemäss des Lastenhefts wurde der Auftrag erteilt, eine Simulationssoftware zu entwickeln. Diese Simulationssoftware soll die Einfügedämpfung eines EMI-Filters simulieren und grafisch darstellen. EMI-Filter werden üblicherweise in Schaltnetzteile verbaut, um zu verhindern, dass Störungen zurück ins Netz gespeist werden. Netzgeräte können unter Umständen hohe Frequenzen erzeugen, die sich nicht gut mit der Netzfrequenz von 50 Hz vertragen. Der EMI-Filter filtert genau diese hochfrequenten Signale heraus, um zu verhindern, dass andere Geräte, die auch ans Netz angeschlossen werden, nicht davon beeinträchtigt werden.

Um die Einfügedämpfung zu ermitteln, soll das EMI-Filter bezüglich der Gleich- und Gegentaktschaltung untersucht werden. Dies geschieht anhand zweier Funktionen für Gleich- und Gegentaktschaltung. Die Funktionen zeigen die Einfügdämpfung für einen Frequenzbereich bis 30MHz.Die beiden Schaltungen beinhalten die parasitären Parameter der elektrischen Komponenten, sodass eine möglichst wahrheitsgetreue Simulation gemacht werden kann. Die entwickelte Software soll die Einfügedämpfung grafisch darstellen.  Die Resultate werden für die Gleich- und Gegentaktschaltung in separaten Funktionen dargestellt. Des Weiteren sollen die elektrischen Komponenten der Schaltungen in der Simulationssoftware variiert werden können.
 
Damit sichergestellt werden kann, dass die Simulationen mathematisch korrekt sind, werden alle Berechnungen zuerst in MATLAB durchgerechnet. Diese Ergebnisse werden mit Simulationen der Simulationssoftware MPLAB mindi verglichen. Des Weiteren wird überprüft, wie die Schaltung vereinfacht werden kann. Dies erfolgt einerseits durch Symmetrien der Schaltung, was dazu führt, dass Komponenten zusammengefasst werden können. Andererseits auch durch Weglassen aufgrund von vernachlässigbarem Einfluss auf die Simulationen. Die Softwarestruktur orientiert sich am gängigen Prinzip der MVC(Model-View-Control). Diese Strukturierung begünstigt einen modularen Aufbau, was die Software einfach erweiterbar macht und zudem eine unkomplizierte Wartung ermöglicht. Des Weiteren wird die Software anhand des Testkonzepts Modul für Modul getestet.

Im Fokus des Fachberichts befindet sich die Software, da das zu entwickelnde Produkt eine Simulationssoftware ist. Der Fachbericht ist nach dem Top-Down-Prinzip aufgebaut. In einem ersten Schritt wird die Software als Ganzes Beschrieben. In den darauffolgenden Kapiteln befinden sich die Dokumentationen der beiden Teile der Software. Die Software wird aufgegliedert in den Teil Benutzeroberfläche und den Teil Ermittlung der Einfügedämpfung. Um den Fachbericht schlank zu gestalten, werden sämtliche theoretische Grundlagen im Anhang platziert. Falls in Kapiteln entsprechende Theorie wichtig ist wird darauf verwiesen.

