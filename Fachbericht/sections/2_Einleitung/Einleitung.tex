\section{Einleitung} \label{sec:einleitung}
Dieser Fachbericht richtet sich primär an die Fachcoaches des Projekts. Um nicht wieder von Null zu beginnen wird vom Wissensstand nach Abgabe des Pflichtenhefts ausgegangen. Sämtliche theoretischen Grundlagen werden im Anhang erläutert, im Text wird jeweils darauf Referenziert

\subsection{Ziel} \label{subsec:ziel}
Unser Auftrag in diesem Projekt war es eine Software zu programmieren, die das Verhalten von so genannten EMI-Filtern simuliert. Diese werden üblicherweise in Schaltnetzteile verbaut, um zu verhindern, dass Störungen zurück ins Netz gespeist werden. Diese Netzgeräte können nämlich unter Umständen hohe Frequenzen erzeugen, die sich nicht gut mit der Netzfrequenz von 50 Hz vertragen. Der EMI-Filter filtert genau diese hochfrequenten Signale heraus, um zu verhindern, dass andere  Geräte, die auch ans Netz angeschlossen werden nicht davon beeinträchtigt sind.\\
Das Tool soll in der Lage sein den Filter in den Schaltungen Differential Mode und Common Mode zu berechnen. Ausserdem soll das User Interface sehr intuitiv und Bedienungsfreundlich sein, damit man schnell zum Resultat kommt. Des Weiteren kann man seine eigegeben Parameter Speichern und ein Filterprofil anlegen, was man später wieder aufrufen kann. Ein weiterer wichtiger Punkt ist, dass das Programm stabil laufen sollte, ohne dass es zu Ausfällen kommt.
\subsection{Vorgehen} \label{subsec:vorgehen}
Zu Beginn der Arbeit steht eine Recherche über die Grundlagen von EMI-Filtern und Java-Programmierstrukturen. Anhand des Lastenhefts des Auftraggebers und der Recherchen wird ein Pflichtenheft mit Lösungsvorschlägen eruiert. Sobald diese vom Auftraggeber abgesegnet werden beginnt die Umsetzung der Arbeit in Form der Software. Die Funktionen und der Aufbau der fertigen Software wird in diesem Fachbericht ausführliche beschrieben. Die Arbeit am Projekt wird von einem Testkonzept begleitet. Dieses beinhaltet verschiedene Tests um Fehler frühzeitig zu erkennen. Das Endprodukt wird im Rahmen einer Schlussräsentation dem Auftraggeber überreicht.

\subsection{Resultate}\label{subsec:resultate}
Als primäres Resultat steht das Hauptprodukt, die Software. Aus dem Vorgehen entstehen weiter die Präsentationen, das Pflichtenheft und der Fachbericht

\subsection{Gliederung} \label{subsec:gliederung}
Im Kapitel Software ist der Ist-Zustand des Produkts beschrieben. Die Software ist analysiert und detailliert beschrieben. Alle Funktionen sind erläutert und die wichtigsten Zusammenhänge werden aufgezeigt. \\Im folgenden Kapitel "Testkonzept" ist beschrieben in welcher Form die parallel laufenden Tests durchgeführt werden. Die Erwartungen an die Tests sind dargelegt und in einer analytischen Validierung sind die Testergebnisse aufbereitet. \\Das abschliessende Kapitel gibt dann ein Fazit über das Produkt und die Arbeit an sich. \\Im Anhang befinden sich sämtliche, zum verständnis relevanten theoretischen Grundlagen und diverse Testergebnisse.


%TODO Einsatzbereich und Anwendung unseres Tools herausfinden, Umschreiben gemäss Feedback

