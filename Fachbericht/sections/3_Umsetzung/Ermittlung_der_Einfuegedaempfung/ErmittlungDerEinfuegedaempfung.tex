\subsection{Ermittlung der Einfügedämpfung} \label{subsec:ermittlung}
In diesem Kapitel wird erläutert, wie von den Schaltungen für Gegentakt und Gleichtakt, die Einfügedämpfung(engl. Insertion Loss, IL) berechnet wird. Unteranderem wird Modular beschrieben, wie die Berechnungen im Model der Software implementiert ist. \\

Die Einfügedämpfung ist eine Grösse, die verwendet wird, um das Verhalten einer Schaltung zu beschreiben. In diesem Fall handelt es sich um einen EMI-Filter. Sie beschreibt das Verhältnis zwischen der eingehenden Leistung zur abgegebenen Leistung. Es handelt sich um eine logarithmische Grösse. Um die Einfügedämpfung in einem Bereich von bis 30MHz abzudecken wird die Formel \ref{equ:Einfügungsverluste} verwendet. In der Formel \ref{equ:Einfügungsverluste} wird die Einfügedämpfung mittels Streuparameter(S-Parameter) berechnet. Die Streuparameter werden in den Grundlagen(Verweis) ausführlich beschrieben. Der Streuparameter $S_{21}$ beschreibt im wesentlichen den Transmissionsgrad des eingehenden Signals. Der Transmissionsgrad beschreibt, welchen Anteil des eingehenden Signals am Ausgang wieder herauskommt.\\

\begin{equation}\label{equ:Einfügungsverluste}
	IL = -20*log (\left\lvert S_{21} \right\rvert)
\end{equation}

Die Einfügedämpfung wird auf Gleichtakt und Gegentakt aufgetrennt. Dieses Prinzip ist eine gängige Methode und wird im Anhang(Verweis) ausführlich beschrieben. Für die beiden Schaltungen, welche der Aufgabenstellung(Verweis Aufgabenstellung) entnommen wurden, werden die S-Parameter berechnet. Damit dies möglich ist, wurden die Schaltungen weitgehend vereinfacht. Dies wird in den Grundlagen(Verweis) Schritt für Schritt beschrieben. Aus den Vereinfachungen gehen folgende Schaltungen hervor(Abbildung \ref{fig:cmschaltungvereinfacht}, \ref{fig:dmschaltungvereinfacht}).

\begin{figure}[H]
		\centering
		\includegraphics[width = 10cm]{EMI_CMvereinfacht.png}
		\label{fig:cmschaltungvereinfacht}
		\caption{reduzierte Gleichtaktschaltung}
\end{figure}

\begin{figure}[H]
		\centering
		\includegraphics[width = 10cm]{EMI_DMvereinfacht.png}
		\label{fig:dmschaltungvereinfacht}
		\caption{reduzierte Gegentaktschaltung}
\end{figure}
\newpage
Um die beiden Schaltungen anhand der Streuparameter zu beschreiben, werden die einzelnen Schaltungsteile anhand von Kettenmatrizen beschrieben(siehe Anhang Verweis Kettenmatrix). Wie im Grundlagen(Verweis Längs- Querimpedanz) beschrieben, werden dazu die Kettenmatrizen für Längsimpedanzen und Querimpedanzen verwendet. Dies führt dazu, dass die einzelnen Schaltungsteile entweder eine Längsimpedanz oder eine Querimpedanz bilden müssen. Die Aufteilung der einzelnen Schaltungsteile in Längs-Querimpedanz und ist in den Abbildungen (Verweis Aufteilung 1 und 2) grafisch dargestellt, wobei „LI“ eine Längsimpedanz kennzeichnet und „QI“ eine Querimpedanz.
\begin{figure}[H]
		\centering
		\includegraphics[width = 10cm]{EMI_CMvereinfacht_markiert.png}
		\label{fig:cmschaltung}
		\caption{Einteilung der Gleichtaktschaltungteile}
\end{figure}

\begin{figure}[H]
		\centering
		\includegraphics[width = 10cm]{EMI_DMvereinfacht_markiert.png}
		\label{fig:dmschaltung}
		\caption{Einteilung der Gegentaktschaltungsteile}
\end{figure}

Im nächsten Schritt werden die Impedanzen der einzelnen Schaltungsteile gebildet, welche in die passenden Kettenmatrizen eingesetzt werden. Um die Kettenmatrizen der Gesamtschaltungen zu bilden, werden die einzelnen Kettenmatrizen durch Kaskadieren zusammengefasst(Verweis Theoretische Grundlagen: Kettenmatrix). Die Kaskadierung entspricht im wesentlichen der Matrixmultiplikation der Kettenmatrizen. \\

\begin{equation}\label{equ:s21ausAMatrix}
s_{21} = \frac{2}{A_{11}+\frac{A{12}}{R_w}+A_{21}*R_w+A_{22}}
\end{equation}

Anhand der Kettenmatrizen der beiden Schaltungen, wird mit der Formel \ref{equ:s21ausAMatrix} der Streuparameter $S_{21}$ berechnet. A1 bis A4 entsprechen den einzelnen Einträge der Matrix. $R_w$ ist die Bezugsimpedanz. Die Bezugsimpedanz bezieht sich auf die Innenimpedanz der Quelle und die Lastimpedanz. Bezüglich der Aufgabenstellung(Verweis Aufgabenstellung) ist dieser auf 50Ohm festgelegt. Ausserdem ist es wichtig, dass die beiden Impedanzen gleich gross sind damit die Schaltung Reziprok ist (Verweis Theoretische Grundlagen: Reziproke Schaltungen). (Wieso ?) 
Bei der reduzierten Gegentaktschaltung (Abbildung \ref{fig:dmschaltung}) werden die beiden Impedanzen mit 25Ohm aufgeführt. Dies ergibt sich durch das Zusammenfassen der Gegentaktschaltung.\\

\subsection{Zusammenfassen der Schaltungen}\label{subsec:zusammenfassungSchaltung}
Folgendes Kapitel zeigt Schritt für Schritt auf, wie die gegebenen Schaltungen vereinfacht werden können. Zudem wird beschrieben welche Komponenten vernachlässigt werden können. Der erste Abschnitt behandelt die Gleichtaktschaltung und in einem zweiten Abschnitt wird die Gegentaktschaltung behandelt.

\paragraph{Gleichtaktschaltung}\label{para:zusammenfassungGleichtakt}
Abbildung \ref{fig:CMSchaltungOriginal} zeigt die Schaltung aus der Aufgabenstellung. 
\begin{figure}[H]
	\centering
	\includegraphics[width = 10cm]{CM_Aufgabenstellung.png}
	\caption{Originale Gleichtaktschaltung\cite{aufgabenstellung}}
	\label{fig:CMSchaltungOriginal}
\end{figure}
Die Originalschaltung ist mit den Komponenten $R_w$ und $L_r$ ergänzt worden, sodass sie symmetrisch ist(siehe Abbildung \ref{fig:CMSchaltungErgänzt}). Dies macht es möglich, dass die Schaltung zur Simulation wie folgt vereinfacht werden kann.
\begin{figure}[H]
	\centering
	\includegraphics[width = 15cm]{EMI_CMpretty1.png}
	\caption{Ergänzte Gleichtaktschaltung}
	\label{fig:CMSchaltungErgänzt}
\end{figure}
Da der obere (siehe Abbildung \ref{fig:CMSchaltungErgänzt}, Nr. 1) und untere Strang (siehe Abbildung \ref{fig:CMSchaltungErgänzt}, Nr. 2) identisch sind und es keinen Potentialunterschied zwischen ihnen gibt, kann die Schaltung, wie folgt, zusammen gefasst werden (siehe Abbildung \ref{fig:CMSchaltungvereinfacht}). Die Schaltung wird entlang der Symmetrie-Achse (siehe Abbildung \ref{fig:CMSchaltungErgänzt}, Nr. 3) aufgetrennt. Somit fallen die Kondensatoren $C_3$, $C_4$, $C_{x1}$ und $C_{x2}$ komplett weg. Die übrigen Komponenten von $L_0$ bilden eine Parallelschaltung, welche sich durch halbieren der Widerstände und Induktivitäten und verdoppeln der Kapazitäten zusammenfassen lässt. Zusätzlich werden die beiden $C_y$ und $C'_{y}$ parallel auf das Bezugspotential geschalten. Da $C_y$ und $C'_y$ identisch sind, werden sie wie in Abbildung \ref{fig:CMSchaltungvereinfacht} zusammengefasst. Diese vereinfachte Schaltung bildet die Grundlage für die Berechnungen der Software.
\begin{figure}[H]
	\centering
	\includegraphics[width = 10cm]{EMI_CMpretty2.png}
	\caption{Vereinfachte Gleichtaktschaltung}
	\label{fig:CMSchaltungvereinfacht}
\end{figure}
\paragraph{Gegentaktschaltung}\label{para:zusammenfassungGegentakt}
Abbildung \ref{fig:DMSchaltungAufgabenstellung}
\begin{figure}[H]
	\centering
	\includegraphics[width = 15cm]{DM_Aufgabenstellung.png}
	\caption{Ergänzte Gegentaktschaltung}
	\label{fig:DMSchaltungAufgabenstellung}
\end{figure}
