\section{Grundlagen} \label{sec:grundlagen}

\subsection{Schaltungsaufbau} \label{subsec:schaltungsaufbau}
(Grundschaltung erklären)
\subsection{Parasitäre Paramter}\label{subsec:parasitparam}
(Ersatzschaltungen von Spule und Kondensator erklären)
\subsection{Gleichtakt} \label{subsec:gleichtakt}
(Gleichtakt erklären und Schaltung zeigen. Vereinfachen der Schaltung schrittweise herleiten)
\subsection{Gegentakt} \label{subsec:gegentakt}
(Gegentakt erklären und Schaltung zeigen. Vereinfachen der Schaltung schrittweise herleiten)
\subsection{Insertion loss/ Streuparameter}\label{subsec:streuparam}
(Insertion loss erklären. Zudem erklären wieso Streuparameter verwendet werden. Was sind Streuparameter?)
Die Einfügungsverluste sind wie folgt definiert: 
\begin{equation}\label{equ:Freqgang}
	IL = \left\lvert H(j\omega) \right\rvert = 20*log(\frac{ \left\lvert U_{20} \right\rvert }{ \left\lvert U_2 \right\rvert })
\end{equation}
In der Definition kann das Spannungsverhältnis durch den Streuparameter \ref{subsec:streuparam} (S-Parameter) S\textsubscript{21} ersetzt werden \ref{equ:Einfügungsverluste}.
\begin{equation}\label{equ:Einfügungsverluste}
	IL = -20*log (\left\lvert S_{21} \right\rvert)
\end{equation}
 Dieser Parameter beschreibt den Transmissionsgrad des Filters. Die Einfügungsverluste wären auch mit dem Verhältnis von eingehende zu abegegebene Leistung zu berechnen, jedoch eignet sich diese Methode mehr beim messtechnischen bestimmen der Einfügungsverluste. 
 
 Die Streuparameter (S-Parameter) werden in der Hochfreqeunztechnik verwendet, um das Verhalten von n-Toren zu beschreiben. Bei einem 2-Tor sind vier Streuparameter von nöten um das Verhalten zu beschreiben. Sie beschreiben die Transmission von Tor 1 zu Tor 2, sowie von Tor 2 zu Tor 1. Des weiteren zeigen sie die Reflexion an den Toren auf. Abbildung \ref{fig:2-Tor} \nameref{fig:2-Tor} zeigt die Streuparameter an einem 2-Tor. 
\begin{figure}[H]
	\centering
	\includegraphics[width=15cm]{s_params_def.png}
	\caption{2-Tor Wellengrössen und Anschlussleitungen \cite{hftech}}
	\label{fig:2-Tor}
\end{figure}
Bei den S-Parameter werden die Eingangs- und Ausgangsgrössen nicht direkt anhand elektrischer Ströme und Spannungen beschrieben. Sie werden mithilfe von Wellengrössen beschrieben, wobei a\textsubscript{i} die einlaufenden Wellen sind und b\textsubscript{i} die Reflektierenden Wellen. Der Index i stellt den Torindex dar. Formel \ref{equ:def_a} und \ref{equ:def_b} zeigen wie die Wellengrössen a\textsubscript{i} sowie b\textsubscript{i} definiert sind.
\begin{equation}\label{equ:def_a}
	a_{ i } = \frac{ U_{ i}+R_{ Wi }I_{ i }}{2*\sqrt{ R_{ Wi } }}
\end{equation}
\begin{equation}\label{equ:def_b}
	b_{ i } = \frac{ U_{ i}-R_{ Wi }I_{ i }}{2*\sqrt{ R_{ Wi } }}
\end{equation}
Die Wellengrössen gelten nur für den gegebenen Bezugswiderstand R\textsubscript{Wi}. Der Bezugswiderstand kann einerseits der Innenwiderstand der angeschlossenen Quelle sein oder der Lastwiderstand der angeschlossenen Last.

Aus der Abbildung 2.3 lässt sich folgende Streumatrix darstellen (Formel \ref{equ:scatteringMatrix}):
\begin{equation}\label{equ:scatteringMatrix}
	\left[
		\begin{matrix}b_1 \\ b_2 \end{matrix}
	\right]
 	=
 	\left[
 		\begin{matrix}
			s_{11}&s_{12} \\s_{21}&s_{22}
		\end{matrix}
	\right]
	* 
	\left[
		\begin{matrix}
			a_1\\b_2
		\end{matrix}
	\right]
\end{equation}
Die Elemente der S-Matrix sind:

\begin{equation}\label{equ:def_s11}
	s_{11} = b_1/a_1\textsf{ Eingangsreflexionsfaktor bei angepasstem Ausgang (a\textsubscript{2}=0)}
\end{equation}
\begin{equation}\label{equ:def_s12}
	s_{12} = b_1/a_2\textsf{ Rückwärtstransmissionsfaktor bei angepasstem Eingang (a\textsubscript{1}=0)}
\end{equation}
\begin{equation}\label{equ:def_s21}
	s_{21} = b_2/a_1\textsf{ Vorwärtstransmissionsfaktor bei angepasstem Ausgang (a\textsubscript{2}=0)}
\end{equation}
\begin{equation}\label{equ:def_s22}
	s_{22} = b_2/a_2\textsf{ Ausgangsreflexionsfaktor bei angepasstem Eingang (a\textsubscript{1}=0)}
\end{equation}
\newpage

\newpage
\subsection{Kettenmatrix}\label{subsec:kettenmatrix}
(Wie kommt man mithilfe der Kettenmatrix auf die Streuparameter. Wie bildet man die Kettenmatrix. Quer und Längsimpedanz genügen. Wie bildet man die Gesamtkettenmatrix der Schaltung)