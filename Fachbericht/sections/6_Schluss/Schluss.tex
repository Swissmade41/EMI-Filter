\section{Schluss} \label{sec:schluss}

Im laufe des Frühlingssemester 2019 hat unser Team an einer Software gearbeitet, die die Einfügedämpfung von EMI-Filtern ,in Abhängigkeit der Frequenz, darstellen kann. In den ersten Wochen wurde ein Pflichtenheft erstellt, um die technischen Grundlagen zu erarbeiten. Zudem wurde auch ein organisatorisches Pflichtenheft geschrieben, um einen sinnvollen Zeiplan festzulegen und die Zusammenarbeit im Team zu regeln. 
Im Rahmen einer Zwischenpräsentation wurden diese Pflichtenhefte und den damaligen Stand der Software vorgestellt. In der Projektwoche wurde die Software fertiggestellt (Version 0.9.5). Ausserdem wurde innerhalb dieser Woche ein Grossteil dieses Fachberichtes geschrieben. Anschliessend wurde die letzte Testphase eingeleutet, um die letzten Fehler in der Software auszumerzen.   
 
Nach den erfolgreich durchgeführten Tests, wird die Software dem Auftraggeber zur Verfügung gestellt. Alle im Pflichtenheft gesteckte Muss-Ziele wurden zufriedenstellend umgesetzt. Die Software ist in der Lage die Eifügedämpfung eines EMI-Filters grafisch darzustellen. Ausserdem ist es möglich verschiedene Filterprofile anzulegen und sie zu speichern. Die verschieden Filter können gleichzeitig im Plot angezeigt werden. 

Die grösste Herausforderung während des Projektes bestand aus der Erarbeitung der elektrotechnischen Grundlagen und die Berechnung der Einfügedämpfung des Netzwerkes in DM und CM. Die Entwicklung des Softwaregrundgerüsts ging zügig vorwärts. Die Schwierigkeiten bei der Software waren beispielsweise die Implementierung der durchgeführten Berechnungen in den Code und die Verarbeitung der Werte, die in die GUI eingegeben werden. Um die Performence nicht einzuschränken, können nur 10 Filterprofile gleichzeitig angezeigt werden.

Schlussendlich konnten alle Schwierigkeiten gut gemeistert werden. Die Software könnte natürlich noch weiterentwickelt werden.  Eine weitere sinnvolle Funktion wäre eine Monte-Carlo Analyse. Diese wurde jedoch aus Zeitgründen nicht implementiert. Als abschliessendes Fazit ist zu sagen, dass dieses Projekt erfolgreich und termingerecht durchgeführt werden konnte. 
