\section{Schluss} \label{sec:schluss}
Zum Abschluss des Projektes und nachdem alle Test durchgeführt wurden, wird die Software dem Auftraggeber zur Verfügung gestellt. Es konnten alle muss-Ziele , die im Pflichtenheft gesteckt wurden zufriedenstellend umgesetzt werden. Die Software ist in der Lage die Eifügedämpfung eines EMI-Filters grafisch darzustellen. Ausserdem ist es möglich verschiedene Filterprofile anzulegen und sie zu speichern. Die verschieden Filter können gleichzeitig im Plot angezeigt werden. 

Die grösste Herausforderung während des Projektes bestand aus der Erarbeitung der elektrotechnischen Grundlagen und die Berechnung der Eifügedämpfung des Netzwerkes in DM und CM. Die Entwicklung des Softwaregrundgerüsts ging zügig vorwärts. Die Schwierigkeiten bei der Software waren Beispielsweise die Implementierung der durchgeführten Berechnungen in den Code und die Verarbeitung der Werte, die in die GUI eingegeben werden. Um die Performence nicht einzuschränken können nur 10 Filterprofile gleichzeitig angezeigt werden.
Schlussendlich konnten alle Schwierigkeiten gut gelöst werden. Die Software könnte natürlich noch weiterentwickelt werden.  Eine weitere sinnvolle Funktion wäre eine Monte-Carlo Analyse. Diese wurde jedoch, aus Zeitgründen nicht implementiert. Als Abschliessendes Fazit ist zu sagen, dass dieses Projekt erfolgreich und Termingerecht durchgeführt werden konnte. 
