\section{Einleitung} \label{sec:einleitung}
Der Einsatz von Schaltnetzteilen kann dazu führen, dass Störungen ins Netz eingespiesen werden. Diese Störungen beeinträchtigen andere mit dem Netz verbundenen Geräte. Um diesen Störungen entgegenzuwirken, werden EMI-Filter eingebaut. Umgekehrt schützen sie die Geräte vor Störungen, die vom Netz zurückfliessen. EMI-Filter werden anhand der Einfügungsdämpfung charakterisiert, welche den Transmissionsgrad des einkommenden Signals beschreibt. Um einen EMI-Filter zu dimensionieren, muss sichergestellt werden, dass die entsprechenden Normen eingehalten werden. Dafür muss der Einfluss von den Komponenten des EMI-Filters auf die Einfügungsdämpfung bekannt sein. 

Damit ein EMI-Filter dimensioniert werden kann, soll eine Simulationssoftware das Verhalten grafisch darstellen. Dies ist das Ziel/ Aufgabe im pro2E, FHNW. Um die Einfügungsdämpfung zu ermitteln, soll der EMI-Filter bezüglich vorgegebener Gleich- und Gegentaktschaltung untersucht werden. Dies geschieht anhand zweier Funktionen für Gleich- und Gegentaktschaltung. Die Funktionen zeigen die Einfügungsdämpfung für einen Frequenzbereich bis 30MHz. Die beiden Schaltungen beinhalten zudem die parasitären Parameter der elektrischen Komponenten, sodass eine realitätsnahe Simulation möglich ist. Des Weiteren soll die entwickelte Software die Einfügungsdämpfung grafisch darstellen. Die Resultate werden für die Gleich- und Gegentaktschaltung in separaten Funktionen dargestellt. Es soll zudem die Möglichkeit bestehen die elektrischen Komponenten der Schaltungen in der Simulationssoftware zu variieren.

Das Entwickeln der Simulationssoftware wird in zwei wesentliche Schritte unterteilt. Der erste Schritt besteht darin, die gegebenen Schaltungen zu vereinfachen, sodass die Berechnungen der Einfügungsdämpfung in der Software implementiert werden können. Im zweiten deutlich umfangreicheren Teil wird die Software programmiert, welche auf dem Prinzip MVC(Model-View-Control) basiert. Es werden Eingabemöglichkeiten implementiert, die es erlauben Komponenten, wie Induktivitäten, Kapazität und Widerstände des EMI-Filter, zu variieren. Zur Visualisierung dienen Funktionen, die die Einfügungsdämpfung in einem weiten Frequenzbereich darstellen.

Der Fokus des Fachberichts liegt auf die zu entwickelte Simulationssoftware. In einem ersten Schritt werden die theoretischen Grundlagen aufgeführt, die zum Entwickeln der Software benötigt werden. Im nächsten Kapitel wird erläutert, wie die zu simulierenden Schaltungen vereinfacht sind, sodass sie berechnet werden können. Außerdem wird darauf eingegangen, wie die Berechnungen in der Software implementiert sind. Im darauffolgenden Teil werden die Komponenten der Benutzeroberfläche aufgeführt und im Detail beschrieben.

