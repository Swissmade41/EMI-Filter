\section{Testkonzept}\label{sec:testkonzept}
Wieso, weshalb und Warum
\subsection{Prinzip} \label{subsec:prinzip}

Unser Testkonzept soll eine einwandfreie Funktion unserer Software garantieren und ein einfache Bedienung ermöglichen. Dies wird durch verschiedene Tests realisiert. 

Grundsätzlich wird alles was implementiert wird, noch einmal sorgfältig durchgelesen und im Fall der Software, wird noch mal kontrolliert, ob keine Fehlermeldungen im Code angezeigt werden. Zudem werden die Java Methoden einzeln ausgeführt, um sich über die Funktion zu vergewissern. Dies erfordert ein hohes Mass an Selbstverantwortung. 
Ausserdem wird so oft wie möglich neu eingefügter Code vom ganzen Team angeschaut, damit alle Teammitglieder die Funktionsweise der Software verstehen.
Im Elektrotechnik-Teil ist es sehr wichtig die ausgerechneten Werte zu hinterfragen und auf ihre Richtigkeit zu überprüfen. Dafür wird das Rechenprogramm MATLAB verwendet, weil es eines der umfangreichsten Mathematikprogramme auf dem Markt ist. Je nach Bedarf werden die Fachcoaches kontaktiert, falls noch Unklarheiten bestehen. Diese Kontinuierliche Tests werden von Anfang an durchgeführt, um Folgefehler zu vermeiden, was uns mehr Zeit für Verbesserungen und neue Features gibt.  
Nach der Projektwoche ist die Version 0.9.5 fertig. Diese wird nach Testprotokoll gründlich getestet. Zunächst wird besprochen, ob man die sich gesteckten Ziele zufriedenstellend erreicht hat. Die Version wird auch dem Auftraggeber Dr. Luca Dalessandro zur Verfügung gestellt, damit er die Möglichkeit hat seine Meinung und Ideen noch einmal einzubringen. Anschliessend beginnt die eigentliche Testphase. Die Software wird zum einen mit einem Kompatibilitätstest gefordert. Bei diesem Test werden Eingaben getätigt, die unsere Software an die Grenzen bringen dürfte. Es werden auch Fehleingaben gemacht, um zu sehen wie die Software darauf reagiert. Ein anderer Aspekt um die Kompatibilität zu prüfen ist, die Verwendung von verschieden Betriebssystemen (Mac OS, WIndows). Um eine saubere Darstellung auf allen Displays zu gewährleisten werden Bildschirme mit unterschiedlichen Auflösungen verwendet (Full HD, 4K).
Danach werden die Tests durch dritte Personen durchgeführt. Dabei werden Experten und auch Fachfremde Tester gesucht. Die Fachfremden erhalten eine grundlegende Einführung über EMI-Filter, damit sie verstehen wofür diese Software entwickelt wurde. Über die Funktionsweise der GUI an sich erhalten die Testpersonen jedoch keine Einführung. Damit kann man sehr gut prüfen, wie einfach und intuitiv die Softwarebedienung ist. Dafür wird das bereits erwähnte Testprotokoll verwendet, in das die Testpersonen ihre Meinung und allfällige Anregungen hineinschreiben können.
Die Fragebogen werden ausgewertet und anschliessend wird sich im Team über allfällige Änderungen ausgetauscht. 
Vor der Abgabe wird die Software noch einem Getestet und ein Abnahmeprotokoll erstellt, bis die Software den Weg zum Auftraggeber findet.
   
\subsection{Erwartungen} \label{subsec:erwartungen}
Zum Testkonzept gehören auch die zu erwartenden Resultate. Damit man eine Vorstellung dafür bekommt, wie die Software auf externe Personen wirkt und man sich besser auf das kommende Feedback einstellen kann. Dies hat den Vorteil, dass man einen Schritt zurück tritt und anders darüber urteilen kann. 
Wir erwarten ,dass die Tester ein umfangreiches Feedback liefern werden. Besonders die Fachpersonen werden höchst wahrscheinlich einiges zu sagen haben bei dem wir dann nicht alles davon umsetzten können. Die Fachfremden Testpersonen werden Mühe haben zu verstehen, um was es überhaupt bei dieser Software geht. Dafür werden die Fachfremden viel über die Bedienungsfreundlichkeit
beifügen, was uns auch sehr weiterhelfen kann.



\subsection{Validierung} \label{subsec:validierung}
Testergebnisse darstellen und Interpretieren

Ebenfalls wird hier beschrieben welche Werte wir mit der Simulation und mit Matlab erreicht haben
