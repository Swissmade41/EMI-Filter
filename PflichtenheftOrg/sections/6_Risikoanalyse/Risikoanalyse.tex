\section{Risikoanalyse}
\begin{figure}[H]
	\centering
	\includegraphics[width=14cm]{Gewichtung_P2.png}
	\label{fig:Gewichtung}
\end{figure}

\newpage



\begin{figure}[H]
\centering
\includegraphics[width=12.5cm]{Risikotabelle_P2.png}
\label{fig:Risikotabell}
\end{figure}

\newpage
\begin{figure}[H]
	\centering
	\includegraphics[width=14cm]{Legende_P2.png}
	\label{fig:Tabelle}
\end{figure}
Um auf Risiken vorbereitet zu sein, wurde obige Risikotabelle erstellt. In dieser listen wir die möglichen Gefahren auf und nennen Präventionsmassnahmen, um sowohl die Eintrittswahrscheinlichkeit(Pi), als auch die Auswirkungen(Si) zu minimieren.\\

%TODO Risiken werden während der Projektplanung ebenfalls eruiert und tabellarischkategorisiert. Dabei werden ebenfalls Folgerisiken und Dringlichkeit beurteilt. Um diesen Risiken entgegenzuwirken werden Präventionsmassnahmen ausgearbeitet und implementiert. 

%TODO Sollte dieser text nicht direkt nach dem titel zu kapitel 6 erscheinen? und danach alle tabellen, etc...


Auf der folgenden Risikomap sind alle Gefahren jeweils mit und ohne Präventionsmassnahme graphisch dargestellt.

\begin{figure}[H]
	\centering
	\includegraphics[width=14cm]{Matrix_verschiebung_P2.png}
	\label{fig:Matrix_verschiebung}
\end{figure}


\begin{figure}[H]
	\centering
	\includegraphics[width=14cm]{Risikouebersicht_P2.png}
	\label{fig:Risikoüebersicht}
\end{figure}
