\section{Einleitung}
\subsection{Ausgangslage}
Im Projekt 2 des Studiengangs Elektro- und Informationstechnik geht es um EMI- Filter, die z.B in ein Netzteil eingebaut werden. Diese Filter verhindern die Einspeisung von Störungen ins Netz.

Schaffner hat uns nun den Auftrag erteilt ein Simulationsprogramm, in Form einer GUI, für Netzwerkfilter zu entwickeln. Die Anforderungen an das Programm sind, dass die Dämpfungseigenschaften des Filters und die Einfügungsverluste ermittelt werden können, sowohl für common mode als auch für differential mode. Die Parameter der parasitären Einflüsse können um ± 30 Prozent verändert werden. Das Programm soll einen analytischen Ansatz verfolgen oder mit einer Simulation gelöst werden.
 
Das Projekt wird durch Niklaus Schwegler geleitet. Bei wöchentlichen Sitzungen, bei denen alle Teammitglieder teilnehmen, wird sich auf den neusten Stand gebracht. Das Team wird grob in 3 Gruppen eingeteilt: Elektrotechnik, Software und Organisation. Die Kommunikation innerhalb des Teams wird über einen Discordserver realisiert.  Als Sammelpunkt für alle Dateien die innerhalb des Projektes erstellt werden auf GIthub hochgeladen, so ist es für alle ersichtlich wer welches Dokument geschrieben hat und was angepasst wurde. Um diese zu erstellen wird LATEX verwendet. Zusetzlich wird die Zeiterfassung mit Hilfe einer Exel-Tabelle gemacht.   



