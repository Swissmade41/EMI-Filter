\section{Einleitung}
\subsection{Ausgangslage}
Im Projekt 2 des Studiengangs Elektro- und Informationstechnik geht es um EMI- Filter, die z.B in ein Netzteil eingebaut werden. Diese Filter verhindern die Einspeisung von Störungen ins Netz.

%Schaffner hat uns nun den Auftrag erteilt ein Simulationsprogramm, in Form einer GUI, für Netzwerkfilter zu entwickeln. Die Anforderungen an das Programm sind, dass die Dämpfungseigenschaften des Filters und die Einfügungsverluste ermittelt werden können, sowohl für common mode als auch für differential mode. Die Parameter der parasitären Einflüsse können um ± 30 Prozent verändert werden. Das Programm soll einen analytischen Ansatz verfolgen oder mit einer Simulation gelöst werden.
 
%Das Projekt wird durch Niklaus Schwegler geleitet. Bei wöchentlichen Sitzungen, bei denen alle Teammitglieder teilnehmen, wird sich auf den neusten Stand gebracht. Das Team wird grob in 3 Gruppen eingeteilt: Elektrotechnik, Software und Organisation. Die Kommunikation innerhalb des Teams wird über einen Discordserver realisiert.  Als Sammelpunkt für alle Dateien die innerhalb des Projektes erstellt werden auf GIthub hochgeladen, so ist es für alle ersichtlich wer welches Dokument geschrieben hat und was angepasst wurde. Um diese zu erstellen wird LATEX verwendet. Zusätzlich wird die Zeiterfassung mit Hilfe einer Excel-Tabelle gemacht.   



Der Auftrag des Auftraggebers, der Schaffner Group lautet, ein Simulationsprogramm für Netzwerkfilter zu entwickeln. Die Anforderungen an die Applikation sind, dass die Dämpfungseigenschaften des Filters simuliert und graphisch angezeigt werden können. Dabei sollen die Gleichtakt- und die Gegentaktstörungen differenziert betrachtet weden können. Ebenfalls soll die Applikation in der Lage sein, die Parameter der parasitären Einflüsse  ± 30 % zu variieren. 
 
Projektleiter ist Niklaus Schwegler. Durch ein externes Dokument werden die Verhaltensregeln innerhalb der Gruppe festgelegt. Darüber hinaus gibt es zusätzliche Übereinkünfte. Bei wöchentlichen Sitzungen, bei denen alle Teammitglieder teilnehmen, wird die Arbeit der  vergangenen Woche analysiert und die kommende Woche geplant. Die Kommunikation läuft über einen Discordserver. Die Datenverwaltung läuft über GitHub, damit für alle ersichtlich ist, wer welche Arbeit eingereicht und bearbeitet hat. Als Schreibprogramm wurde LaTeX gewählt. Die Zeiterfassung wird ebenfalls auf GitHub in einem Excel-File geführt.




