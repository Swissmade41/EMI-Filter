\section{Einleitung}
\subsection{Ausgangslage}
Im Projekt 2 des Studiengangs Elektro- und Informationstechnik geht es um EMI- Filter, die z.B in ein Netzteil eingebaut werden. Diese Filter verhindern die Einspeisung von Störungen ins Netz.

Der Auftrag von Dr. Dalessandro lautet eine Applikation zu entwickeln, welche in der Entwicklung von solchen Filtern eingesetzt werden kann.Die Anforderungen an diese Applikation sind, dass die Dämpfungseigenschaften des Filters ermittelt werden können. Dabei sollen die Gleichtakt- und die Gegentaktstörungen differenziert betrachtet werden können. Ebenfalls soll die Applikation in der Lage sein, die parasitären Einflüsse der einzelnen Parameter (Bauteile) um ± 30 Prozent zu variieren. Das Applikation soll einen analytischen Ansatz verfolgen oder mit einer Simulation gelöst werden.
 
Projektleiter ist Niklaus Schwegler. Bei wöchentlichen Sitzungen, bei denen alle Teammitglieder teilnehmen, wird sich auf den neusten Stand gebracht und die kommende Woche geplant. Durch ein externes Dokument werden die Verhaltensregeln innerhalb der Gruppe festgelegt. Das Team wird grob in 3 Gruppen eingeteilt: Elektrotechnik, Software und Organisation.  Die Datenverwaltung läuft über GitHub, damit für alle ersichtlich ist, wer welche Arbeit eingereicht und bearbeitet hat. Um diese zu erstellen wird LaTeX verwendet.Die Kommunikation innerhalb des Teams wird über einen Discordserver realisiert.  Die Zeiterfassung wird ebenfalls auf GitHub in einem Excel-File geführt.
    







