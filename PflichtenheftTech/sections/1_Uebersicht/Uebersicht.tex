\section{Übersicht} \label{sec:uebersicht}
\subsection{Ausgangslage}

In der modernen Gesellschaft hängen von Jahr zu Jahr mehr elektrische Verbraucher am Stromnetz. Der stetig steigende Leistungsbedarf dieser Verbraucher führt dazu, dass ihre Versorgung angepasst werden muss. Aus dem konventionellen Trafo-Netzteil entstand das sogenannte Schaltnetzteil. Dieses hat grosse Vorteile gegenüber dem Trafo-Netzteil, sowohl wirtschaftlich gesehen, als auch leistungsbezogen. \\Allerdings haben sie auch einen Nachteil. Sie lassen hochfrequente Störungen (EMI), entstehen, welche ins Netz zurückfliessen. Diese Störungen, welche man als Gleichtakt- und Gegentaktrauschen bezeichnet, können wiederum in anderen Verbrauchern  Störungen verursachen. \\Aufgrund dieses Problems wurden verschiedene Normen an Geräte gestellt um diese Emissionen zu minimieren. Daher werden in moderne Schaltnetzteile EMI-Filter verbaut.  Diese, auch Netzfilter genannt, bestehen aus einem Netzwerk von  aus Widerständen, Kondensatoren und Spulen. Da im Schaltnetzteil die Netzfrequenz in eine hochfrequente Spannung gewandlet wird, reagieren die Bauteile als Impedanzen und sie filtern, je nach Bauform, verschiedene hochfrequente Signale aus der Rückspeisung.


Der Auftrag von Dr. Dalessandro lautet eine Applikation zu entwickeln, welche in der Entwicklung von solchen Filtern eingesetzt werden kann. Die Anforderungen an die Applikation sind, dass die Dämpfungseigenschaften des Filters simuliert und graphisch angezeigt werden können. Dabei sollen die Gleichtakt- und die Gegentaktstörungen differenziert betrachtet weden können. Ebenfalls soll die Applikation in der Lage sein, die parasitären Einflüsse der einzelnen Parameter (Bauteile) um ± 30 \% zu variieren.   


Dieses Pflichtenheft beschreibt die technischen Aspekte des Auftrags und liefert bereits Ideen betreffend der Umsetzung. 
 

\newpage
\subsection{Projektziele} \label{subsec:projektziele}



Folge Ziele wurden festgelegt:
\newcommand{\HY}{\hyphenpenalty = 25\exhyphenpenalty = 25}
\begin{table}[H]
\small
\begin{tabular}{>{\HY\RaggedRight}p{5cm} >{\HY\RaggedRight}p{9.5cm} }
\hline
\textbf{Ziel}					&\textbf{Beschreibung}			\\						
\hline
\rowcolor{hellgrau}
\multicolumn{2}{l}{\textbf{1.Elektrotechnik}}\\
1.1 Frequenzverhalten		&\ Verhalten des Rauschens in den Frequenzbereichen von 0-500kHz, von 500kHz-5MHz und von 5MHz-30MHz.\\
1.2 Eingangsdämpfung &\ (Insertion loss)  Berechnung der Grösse der Eingangsdämpfung, dargestellt in einer Kurve. \\
1.3 Parasitäre Parameter &\ (Empfindlichkeitsanalyse) Erkennen und berechnen der Einflüsse der einzelnen Bauteile (Parameter).\\
1.4 Filterdesign &\ Analyse der Schaltung um diese zu Vereinfachen. Wenn möglich auf regelmässige Glieder.\\
\rowcolor{hellgrau}
\multicolumn{2}{l}{\textbf{2. Software}}\\
2.1 Plattform &\  Die Plattform soll möglichst unabhängig von Betriebssystemen gewählt werden. Daher ist JAVA die präferierte Wahl.\\
2.2 Struktur &\	Um die Applikation möglichst erweiterbar zu halten ist das MVC-Framework eine gute Grundstruktur.\\
2.3 GUI &\ Die Gui soll vorallem die Eingangsdämpfung Anzeigen. Ebenfalls sehr attraktiv ist es, wenn die Parameter via Schieberegler direkt verstellt werden können.\\
2.4 Dateneingabe &\ Die Applikation soll möglichst vor Eingabefehlern schützen. Um die Bedienung noch weiter zu erleichtern erhält sie zudem noch einen Speicher um voreingestellte Filter zu speichern und wieder zu laden.\\
2.5 Datenausgabe &\ Der Plot soll logarithmisch dargestellt werden. Zusätzlich soll er auch exportiert werden können.\\	
\rowcolor{hellgrau}
\multicolumn{2}{l}{\textbf{3. Wunschziele}}\\			
3.1 Verschiedene Filter &\ Um einen Mehrwert zu schaffen wäre es Praktisch direkt mehrere Layouts zu vergleichen.\\
3.2 Monte Carlo Simulation &\ Falls die Zeit reicht wäre die Implementation einer Monte Carlo Simulation noch sinvoll.\\	
3.3 Diverse Bedienungshilfen &\ Falls die Zeit reicht können noch diverse Bedienungshilfen und Gadgets eingebaut werden.\\
					
\hline
\end{tabular}
\end{table}

\newpage
\subsection{Lieferobjekte} \label{subsec:lieferobjekt}

Nachfolgend werden alle Lieferobjekte aufgelistet:

\newcommand{\HE}{\hyphenpenalty = 25\exhyphenpenalty = 25}
\begin{table}[H]
\small
\begin{tabular}{>{\HE\RaggedRight}p{5.5cm} >{\HE\RaggedRight}p{4cm} }
\hline
\rowcolor{hellgrau}
\textbf{Beschreibung}					&\textbf{Datum}			\\						
\hline
%\rowcolor{hellgrau}
%\multicolumn{2}{l}{\textbf{Objekt} } {\textbf{Datum}}\\
1. Organisatorisches Pflichtenheft		&\ 24.03.19 und 04.04.19\\
2. Technisches Pflichtenheft		&\ 24.03.19 und 04.04.19\\
3. Disposition und Einleitung		&\ 02.05.19\\
4. Fachbericht	&\ 13.06.19\\
\hline
\end{tabular}
\end{table}
