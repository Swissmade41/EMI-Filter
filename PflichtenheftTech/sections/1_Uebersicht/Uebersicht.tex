\section{Übersicht} \label{sec:uebersicht}
\subsection{Ausgangslage}


In einer hochtechnisierten Gesellschaft, wo pro Haushalt diverse Elektrogeräte zum Einsatz kommen. Stellt sich die Frage, funktioniert das einwandfrei wenn verschiedene Geräte an dasselbe Netz angeschlossen werden?
Niemand will darauf verzichten sich gleichzeitig die Haare zu föhnen und fernzusehen und darum muss eine Lösung her. Um den reibungslosen Betrieb erwähnter Geräte zu gewährleisten, braucht es elektrische Filter. Diese sogenannten Netzfilter schützen die anderen Geräte vor hochfrequenten Emissionen, die im innern eines Gerätes entstehen. Unter einem Netzfilter kann man sich ein Bauteil vorstellen, das am Eingang eines Gerätes geschaltet wird und nur den gewünschten Frequenzbereich, in der Schweiz 50 Hz, durchlässt. 
So weit so gut. Doch wie werden solche Filter denn gemacht?
Die Antwort liegt nahe. Es wird mit Widerständen, Kapazitäten und Induktivitäten ein Netzwerk gebaut dem ein Diagramm beigelegt wird. In diesem Diagramm wird die Dämpfungseigenschaft in Dezibel in Abhängigkeit von der Frequenz aufgezeigt. Ausserdem ist entscheidend welche Einfügungsverluste der Filter hat, das sind die Leistungsverluste im Filter. 
Diese Werte sind matchentscheidend für das bestehen auf einem hochspezialisierten Markt. 
Nun hat Herr Dalessandro von der Schaffner Gruppe uns die Aufgabe gestellt ein tool zu programmieren, dass in der Entwicklung von solchen Filter eingesetzt werden kann. Es geht darum die oben genannten Spezifikationen bereits auf dem Papier aufzuzeigen. Mit diesem einzigartigen Programm kann man nämlich bereits während der Planung eines Filters simlieren, welchen Einfluss die verschiedenen realen Bauteile auf den fertigen Filter haben.
Im Folgenden ein Versuch zur Lösung einer hochkomplexen Aufgabe. 
 

\newpage
\subsection{Projektziele} \label{subsec:projektziele}



Folge Ziele wurden festgelegt:
\newcommand{\HY}{\hyphenpenalty = 25\exhyphenpenalty = 25}
\begin{table}[H]
\small
\begin{tabular}{>{\HY\RaggedRight}p{5cm} >{\HY\RaggedRight}p{6.5cm} >{\HY\RaggedRight}p{3cm}}
\hline
\textbf{Zielkriterium}					&\textbf{Zielvariable}									&\textbf{Randbedingung}\\
\hline
\rowcolor{hellgrau}
\multicolumn{3}{l}{\textbf{1. Elektrotechnik}}\\
- Frequenzverhalten von DM und CM des EMI-Filters		&x		&x\\
- Einfügungsverluste von DM und CM (Insertion loss)\\
- Auswirkung parasitärer Parameter (Empfindlichkeitsanalyse)\\
- Variieren der Parasitäre Parameter im Bereich +-30%\\
- Direkter Vergleich von 2 Filterdesigns bezüglich Leistung\\
3 Frequenzbereiche\\
1.2. dummy\\
\rowcolor{hellgrau}
\multicolumn{3}{l}{\textbf{2. Software}}\\
Plattformunabhängige Software -> Java\\
Gute Softwarestrukturierung\\
Benuztzerfreundlich\\
Einfache Bedienung\\
Geschützt vor Fehleingaben\\
CM/DM Plot in einem eigenen Fenster gross darstellen\\	
\rowcolor{hellgrau}
\multicolumn{3}{l}{\textbf{3. Wunschziele}}\\			
Mehrere Filter gleichzeitig berechne und anzeigen lassen\\
    In einer Tabelle können einzelne Filterprofile verwaltet werden\\
    Die Filterprofile können benennt werden\\
    Die Farbe der Kurve kann verändert werden\\
Filterprofile speichern und laden\\
Monte Carlo\\								
\hline
\end{tabular}
\end{table}

\newpage
\subsection{Nicht-Ziele}
dummytext

Folgende Nicht-Ziele wurden definiert:
\begin{table}[H]
\small
\begin{tabular}{ll}
\hline
\textbf{Nicht-Zielkriterium}				&\textbf{Nicht-Zielvariable}											\\
\hline
\rowcolor{hellgrau}
\textbf{1. Planung}						&	dummy																\\
										&dummy											\\
										&dummy								\\	
\rowcolor{hellgrau}
\textbf{2. Realisierung}					&																	\\
										&dummy								\\
\hline
\end{tabular}
\end{table}

\subsection{Lieferobjekt} \label{subsec:lieferobjekt}

\subsection{Rahmenbedingungen} \label{subsec:rahmenbedingungen}