\section{Übersicht} \label{sec:uebersicht}
\subsection{Ausgangslage}

In der modernen Gesellschaft hängen von Jahr zu Jahr mehr elektrische Verbraucher am Stromnetz. Der stetig steigende Leistungsbedarf dieser Verbraucher führt dazu, dass ihre Versorgung angepasst werden muss. Aus dem konventionellen Trafo-Netzteil entstand das sogenannte Schaltnetzteil. Dieses hat grosse Vorteile gegenüber dem Trafo-Netzteil, sowohl wirtschaftlich gesehen, als auch leistungsbezogen. \\Allerdings haben sie auch einen Nachteil. Sie lassen hochfrequente Störungen (EMI), entstehen, welche ins Netz zurückfliessen. Diese Störungen, welche man als Gleichtakt- und Gegentaktrauschen bezeichnet, können wiederum in anderen Verbrauchern  Störungen verursachen. \\Aufgrund dieses Problems wurden verschiedene Normen an Geräte gestellt um diese Emissionen zu minimieren. Daher werden in moderne Schaltnetzteile EMI-Filter verbaut.  Diese, auch Netzfilter genannt, bestehen aus einem Netzwerk von  aus Widerständen, Kondensatoren und Spulen. Da im Schaltnetzteil die Netzfrequenz in eine hochfrequente Spannung gewandlet wird, reagieren die Bauteile als Impedanzen und sie filtern, je nach Bauform, verschiedene hochfrequente Signale aus der Rückspeisung.


Der Auftrag von Dr. Dalessandro lautet eine Applikation zu entwickeln, welche in der Entwicklung von solchen Filtern eingesetzt werden kann. Die Anforderungen an die Applikation sind, dass die Dämpfungseigenschaften des Filters simuliert und graphisch angezeigt werden können. Dabei sollen die Gleichtakt- und die Gegentaktstörungen differenziert betrachtet weden können. Ebenfalls soll die Applikation in der Lage sein, die parasitären Einflüsse der einzelnen Parameter (Bauteile) um ± 30 \% zu variieren.   


Dieses Pflichtenheft beschreibt die technischen Aspekte des Auftrags und liefert bereits Ideen betreffend der Umsetzung. 
 

\newpage
\subsection{Projektziele} \label{subsec:projektziele}


In der Folgenden Tabelle(ref?) werden alle Ziele aufgeleistet.In Kapitel 2 werden sie ausformuliert und ihre Implementation wird erläutert. 
\newcommand{\HY}{\hyphenpenalty = 25\exhyphenpenalty = 25}
\begin{table}[H]\label{tab:ziele}\caption{Ziele}
\small
\begin{tabular}{>{\HY\RaggedRight}p{7cm} >{\HY\RaggedRight}p{1.5cm} >{\HY\RaggedRight}p{1.5cm} >{\HY\RaggedRight}p{3cm}}
\hline
\textbf{Ziel}					&\textbf{Muss}	&\textbf{Soll}	&\textbf{Zielbezeichnung}			\\						
\hline
\rowcolor{hellgrau}
\multicolumn{4}{l}{\textbf{Fachliche Anforderung}}\\
Unabhängigkeit von Betriebssystemen		&\ x &\  &\ F1\\
Modular aufgebaut/erweiterbar		&\ x &\  &\ F2\\
Berechnungen und GUI getrennt		&\ x &\  &\ F3\\
 Berechnungszeit < 10 Sek.		&\ x &\  &\ F4\\
Unabhängige Komponenten		&\   &\ x &\ F5\\
Verstellbare Parameter		&\ x &\   &\ F6\\
Verschiedene Berechnungsmodi		&\ x &\   &\ F7\\	
Monte-Carlo Simulation &\   &\ x &\ F8\\

\rowcolor{hellgrau}
\multicolumn{4}{l}{\textbf{3. Graphische Anforderungen}}\\			
Visualisierung der Schaltung		&\  &\ x &\ G1\\	
Einfache und schnelle Eingabemöglichkeit &\ x &\  &\ G2\\
Darstellung im Frequenzbereich bis 30MHz		&\ x &\  &\ G3\\
Mehrere Plots gleichzeitig		&\ x &\  &\ G4\\
Auslagern des Plots		&\   &\ x &\ G5\\


\rowcolor{hellgrau}
\multicolumn{4}{l}{\textbf{Anforderungen an die Bedienung}}\\			
Schutz vor Fehleingaben		&\ x &\   &\ B1\\
Export von Plots		&\  &\ x &\ B2\\
Speicherverwaltung		&\   &\ x &\ B3\\
Import von Daten		&\   &\ x &\ B4\\	
				
\hline
\end{tabular}
\end{table}

\newpage
\subsection{Lieferobjekte} \label{subsec:lieferobjekt}

Nachfolgend werden alle Lieferobjekte aufgelistet:

\newcommand{\HE}{\hyphenpenalty = 25\exhyphenpenalty = 25}
\begin{table}[H]\label{tab:ziele}\caption{Lieferobjekte}
\small
\begin{tabular}{>{\HE\RaggedRight}p{5.5cm} >{\HE\RaggedRight}p{4cm} }
\hline
\rowcolor{hellgrau}
\textbf{Beschreibung}					&\textbf{Datum}			\\						
\hline
%\rowcolor{hellgrau}
%\multicolumn{2}{l}{\textbf{Objekt} } {\textbf{Datum}}\\
1. Organisatorisches Pflichtenheft		&\ 24.03.19 und 04.04.19\\
2. Technisches Pflichtenheft		&\ 24.03.19 und 04.04.19\\
3. Beta-Version des GUI	&\ 18.04.19\\
4. Beta-Version der Software &\ 28.04.19\\
5. Fachbericht	&\ 13.06.19\\
\hline
\end{tabular}
\end{table}
