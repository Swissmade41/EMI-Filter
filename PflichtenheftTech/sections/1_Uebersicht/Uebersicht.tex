\section{Übersicht} \label{sec:uebersicht}
\subsection{Ausgangslage}


In einer hochtechnisierten Gesellschaft, wo pro Haushalt diverse Elektrogeräte zum Einsatz kommen. Stellt sich die Frage, funktioniert das einwandfrei wenn verschiedene Geräte an dasselbe Netz angeschlossen werden?
Niemand will darauf verzichten sich gleichzeitig die Haare zu föhnen und fernzusehen und darum muss eine Lösung her. \\ Um den reibungslosen Betrieb erwähnter Geräte zu gewährleisten, braucht es elektrische Filter. Diese sogenannten Netzfilter schützen die anderen Geräte vor hochfrequenten Emissionen, die im innern eines Gerätes entstehen. Unter einem Netzfilter kann man sich ein Bauteil vorstellen, das am Eingang eines Gerätes geschaltet wird und nur den gewünschten Frequenzbereich, in der Schweiz 50 Hz, durchlässt. \\ So weit so gut. Doch wie werden solche Filter denn gemacht? \\
Die Antwort liegt nahe. Es wird mit Widerständen, Kapazitäten und Induktivitäten ein Netzwerk gebaut dem ein Diagramm beigelegt wird. In diesem Diagramm wird die Dämpfungseigenschaft in Dezibel in Abhängigkeit von der Frequenz aufgezeigt. Ausserdem ist entscheidend welche Einfügungsverluste der Filter hat, das sind die Leistungsverluste im Filter. \\
Diese Werte sind matchentscheidend für das bestehen auf einem hochspezialisierten Markt. 
Nun hat Herr Dalessandro von der Schaffner Gruppe uns die Aufgabe gestellt ein tool zu programmieren, dass in der Entwicklung von solchen Filter eingesetzt werden kann. Es geht darum die oben genannten Spezifikationen bereits auf dem Papier aufzuzeigen. Mit diesem einzigartigen Programm kann man nämlich bereits während der Planung eines Filters simlieren, welchen Einfluss die verschiedenen realen Bauteile auf den fertigen Filter haben.
Im Folgenden ein Versuch zur Lösung einer hochkomplexen Aufgabe. 
 

\newpage
\subsection{Projektziele} \label{subsec:projektziele}



Folge Ziele wurden festgelegt:
\newcommand{\HY}{\hyphenpenalty = 25\exhyphenpenalty = 25}
\begin{table}[H]
\small
\begin{tabular}{>{\HY\RaggedRight}p{5cm} >{\HY\RaggedRight}p{9.5cm} }
\hline
\textbf{Ziel}					&\textbf{Beschreibung}			\\						
\hline
\rowcolor{hellgrau}
\multicolumn{2}{l}{\textbf{1.Elektrotechnik}}\\
1.1 Frequenzverhalten		&\ Verhalten des Rauschens in den Frequenzbereichen von 0-500kHz, von 500kHz-5MHz und von 5MHz-30MHz.\\
1.2 Eingangsdämpfung &\ (Insertion loss)  Berechnung der Grösse der Eingangsdämpfung, dargestellt in einer Kurve. \\
1.3 Parasitäre Parameter &\ (Empfindlichkeitsanalyse) Erkennen und berechnen der Einflüsse der einzelnen Bauteile (Parameter).\\
1.4 Filterdesign &\ Analyse der Schaltung um diese zu Vereinfachen. Wenn möglich auf regelmässige Glieder.\\
\rowcolor{hellgrau}
\multicolumn{2}{l}{\textbf{2. Software}}\\
2.1 Plattform &\  Die Plattform soll möglichst unabhängig von Betriebssystemen gewählt werden. Daher ist JAVA die präferierte Wahl.\\
2.2 Struktur &\	Um die Applikation möglichst erweiterbar zu halten ist das MVC-Framework eine gute Grundstruktur.\\
2.3 GUI &\ Die Gui soll vorallem die Eingangsdämpfung Anzeigen. Ebenfalls sehr attraktiv ist es, wenn die Parameter via Schieberegler direkt verstellt werden können.\\
2.4 Dateneingabe &\ Die Applikation soll möglichst vor Eingabefehlern schützen. Um die Bedienung noch weiter zu erleichtern erhält sie zudem noch einen Speicher um voreingestellte Filter zu speichern und wieder zu laden.\\
2.5 Datenausgabe &\ Der Plot soll logarithmisch dargestellt werden. Zusätzlich soll er auch exportiert werden können.\\	
\rowcolor{hellgrau}
\multicolumn{2}{l}{\textbf{3. Wunschziele}}\\			
3.1 Verschiedene Filter &\ Um einen Mehrwert zu schaffen wäre es Praktisch direkt mehrere Layouts zu vergleichen.\\
3.2 Monte Carlo Simulation &\ Falls die Zeit reicht wäre die Implementation einer Monte Carlo Simulation noch sinvoll.\\	
3.3 Diverse Bedienungshilfen &\ Falls die Zeit reicht können noch diverse Bedienungshilfen und Gadgets eingebaut werden.\\
					
\hline
\end{tabular}
\end{table}

\newpage
\subsection{Lieferobjekte} \label{subsec:lieferobjekt}

Nachfolgend werden alle Lieferobjekte aufgelistet:

\newcommand{\HE}{\hyphenpenalty = 25\exhyphenpenalty = 25}
\begin{table}[H]
\small
\begin{tabular}{>{\HE\RaggedRight}p{5.5cm} >{\HE\RaggedRight}p{4cm} }
\hline
\rowcolor{hellgrau}
\textbf{Beschreibung}					&\textbf{Datum}			\\						
\hline
%\rowcolor{hellgrau}
%\multicolumn{2}{l}{\textbf{Objekt} } {\textbf{Datum}}\\
1. Organisatorisches Pflichtenheft		&\ 24.03.19 und 04.04.19\\
2. Technisches Pflichtenheft		&\ 24.03.19 und 04.04.19\\
3. Disposition und Einleitung		&\ 02.05.19\\
4. Fachbericht	&\ 13.06.19\\
\hline
\end{tabular}
\end{table}
