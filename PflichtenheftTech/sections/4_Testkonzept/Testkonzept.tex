\section{Testkonzept} \label{sec:Testkonzept}

Das Testkonzept definiert und beschreibt den Ablauf und die Methodik wie die Software getestet und überprüft wird. Ziel ist eine einwandfreie Funktion sicherzustellen, wie auch die Erfüllung des Auftrages zu überprüfen. Als Wunschziel soll zudem die Benutzerfreundlichkeit optimiert werden.


\subsection{Kontinuierliche Kontrolle } \label{subsec:1}

Bereits während der Entwicklungsphase haben wir die Software kontinuierlich getestet um Folgefehler zu vermeiden. Diese werden jeweils direkt korrigiert und in einem Protokoll dokumentiert. 

\subsubsection{Überprüfung mittels Matlab  } \label{subsubsec:1}

Die Rückgabewerte werden nach jeder Phase mittels Matlab kontrolliert und verglichen.
Dies erfolgt aufbauend, also werden als erstes die einzelnen Klassen und Methoden nachgerechnet und verglichen. Alles wird vor der Integrierung einzeln überprüft, um wiederum Folgefehler zu vermeiden. Anschliessend werden die erstellten Funktionen einzeln überprüft. Schlussendlich wird die Software als Gesamtes durchgerechnet.


\subsection{Softwaretest Intern} \label{subsec:2}

Nach Vollendung der Version 0.9.5 wird die Software auf Herz und Nieren getestet.

\subsubsection{Zielüberprüfung} \label{subsubsec:1}

Die Software wird auf die Erfüllung aller gesteckten Ziele überprüft. Eventuelle Fehler werden behoben und dokumentiert.

\subsubsection{Eingabefehlertest } \label{subsubsec:2}
 
Um eine hohe Qualität zu gewährleisten, werden bewusst Eingaben getätigt, welche die Software an Ihr Limit bringen sollte, um die Stabilität zu überprüfen. 

\subsubsection{Kompatibilität} \label{subsubsec:2}
 
Die Software wird auf Kompatibilität kontrolliert, da die Software auf den verschiedenen Betriebsystem und Bildschirmen funktionieren sollte. Daher werden die Tests sowohl auf  einem Apple OS wie auch auf einem Microsoft Windows der neusten Generation durchgeführt.

Um eine saubere Darstellung bei variierenden Displays sicher zu stellen, werden die Tests zudem auf verschiedenen Displays (falls verfügbar bis 4K) durchgeführt.

\subsection{Überprüfung durch Dritte} \label{subsec:4}

Die Software wird durch Dritte überprüft. Die Prüfer sollen sowohl Fachfremde wie auch Experten sein. Im Vordergrund steht hierbei die intuitive Bedienung zu überprüfen.

\subsubsection{Ablauf} \label{subsubsec:1}

Die Prüfer erhalten eine kurze Einführung über EMI-Filter, die Software Bedienung wird jedoch nicht erklärt. Die Prüfer werden anhand eines Fragebogens ihre Beurteilung abgeben.

\subsubsection{Auswertung} \label{subsubsec:2}

Die  Fragebogen  werden ausgewertet und  in einem Dokument zussamengetragen, im Team wird dann über mögliche Änderungen entschieden 

\subsection{Abnahme durch den Auftraggebers} \label{subsec:4}

Wie besprochen wird Auftraggeber die Software zweimal zu überprüfen.

\subsubsection{Vor-Abnahme} \label{subsubsec:1}

Der Auftraggeber erhält in der Woche 20 die Version 0.9.5 zur Überprüfung. Dies gibt ihm die Möglichkeit seine Änderungswünsche einfliessen zu lassen. Wie in der Überprüfung durch Dritte erhält er keine Einführung in die Bedienung der Software.

\subsubsection{Schlussabnahme} \label{subsubsec:2}

Am Ende des Projekts wird Auftraggeber die Software nochmals überprüfen. Gemeinsam wird ein Abnahmeprotokoll erstellt.